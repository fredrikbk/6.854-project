% Short paragraph to tie intro to abstract
The primal-dual technique was successfully applied to approximate NP-Hard problems in the 1990's.
% Citations needed
In the early 2000's an application of this technique to online algorithms was discovered, and has since then been successfully used to solve many online algorithms.
% Citations needed

% Talk about how it allows you to specify online programs as offline linear programs that are then solved incrementally online using the technique.
% Solve program incrementally by successively adding in variables and constraints
The primal-dual method provides us with a fairly general framework to turn linear programs solving \emph{offline} problems into approximate solvers for the \emph{online} counter-parts.
In online problems information is revealed in pieces, and we must at each step make irrevocable decisions based on the limited knowledge we have thus far.
Correspondingly, the primal-dual method allows us to add one or more constraints to our linear program at a time, and it approximates the solution at each step without changing any assignments to variables it made in the past.
Thus, we do not have to start from scratch when designing an online algorithm that has a known offline linear program.
We can instead start from the linear program, and use the ideas from the primal-dual literature to devise an online algorithm.

% Buchbinder and Joseph's work
In 2009 Buchbinder and Naor published a survey of the application of the primal-dual method to online algorithms~\cite{buchbinder09:survey}.
The survey gave a thorough description of the technique, and went on to survey several papers showing how the primal-dual method can be applied to solve a number of important online problems. 
Examples include online set-cover, caching, routing and ad-auction revenue maximization.

Since then the primal-dual method has been applied to solve additional problems in an online setting.
In this paper we survey three new application of the technique to very different problems.
This demonstrates the power and applicability of the primal-dual approach to online problems.

In section~\ref{primal-dual} we offer an introduction to the primal-dual method using ski rental as a running example.
In section~\ref{steiner-tree} we survey Naor, Panigrahi and Singh's paper from 2011, which presents the first online algorithm for the node-weighted steiner tree problem~\cite{naor11:node-weighted-steiner-tree}.
Section~\ref{k-server} surveys Bansal, Buchbinder and Naor's 2010 paper, which demonstrates the power of the primal-dual approach by using it to explore promising approaches to the online k-server problem: the holy grail of online algorithms~\cite{bansal10:k-server}.
Finally, in section~\ref{job-migration} we survey Buchbinder, Jain and Menache's 2011 paper and companion technical report, in which they design an algorithm for online job-migration between geographically distributed cloud data centers to reduce electricity costs~\cite{buchbinder11:job-migration,buchbinder11:job-migration-techreport}.
This paper demonstrates the flexibility of the primal-dual technique, by applying it to the design of algorithms for complex real-world problems that out-perform reasonable greedy heuristics both in theory and in practice.
