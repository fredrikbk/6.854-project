\subsection{Preliminaries}
In addition to the standard theorems and definitions about linear programs and duality we introduce \emph{approximate} complementary slackness.
First, we review the concept of complementary slackness.
Consider the following primal/dual linear programs:
\[
\textrm{($P$) : min}  \sum^n_{i=1} c_i x_i
\]
\[
	\begin{array}{lr}
	\textrm{subject to :} & \\
	\forall j \textrm{ :} & \sum_{i=1}^n a_{ij} x_i \geq b_j \\
	\forall i \textrm{ :} & x_i \geq 0 \\
	\end{array}
\]

\vspace{0.1cm}
\hrule
\vspace{0.1cm}

\[
\textrm{($D$) : max}  \sum^m_{j=1} b_j y_j
\]
\[
	\begin{array}{lr}
	\textrm{subject to :} & \\	
	\forall i \textrm{ :} & \sum_{j=1}^m a_{ij} y_j \leq c_i \\
	\forall j \textrm{ :} & y_j \geq 0 \\
	\end{array}
\]

A feasible solution $x = (x_i, \ldots, x_n)$ to the primal and a feasible solution $y = (y_i, \ldots, y_m)$ to the dual satisfy the complementary slackness condition if and only if
\begin{itemize}
\item for each $i$ we have either $x_i = 0$ or $\sum_{j} a_{ij}y_j = c_i$
\item and for each $j$ we have either $y_j = 0$ or $\sum_{i} a_{ij} x_i = b_j$.
\end{itemize}
So each variable is either equal to 0 or the corresponding constraint is tight.
A pair of feasible solutions $x$ and $y$ is optimal if and only if it satisifies the complementary slackness condition.

Approximate slackness relaxes the tightness conditions above.
As a result, we only get an approximation ratio between the primal and dual solutions satisfying the approximate complementary slackness condition.
As we will see in the context of online algorithms, this approximation ratio corresponds to the competitive ratio.
The reason is that we formulate an LP for the offline version of a problem which we then approximate with an online algorithm.
So the ratio between the primal and the dual solution gives a bound on the competitive ratio.

Now we precisely state the approximate complementary slackness condition.
Let $x$ and $y$ be feasible solutions to the primal and dual linear programs.
First we define the primal and dual complementary slackness conditions.
\begin{itemize}
\item A feasible solution $x$ satisfies the \emph{primal} complementary slackness condition if for $\alpha \geq 1$ we have either $x_i > 0$ or $c_i / \alpha \leq \sum_j a_{ij} y_j \leq c_i$ for any $i$.
\item Similarly, $y$ satisifies the \emph{dual} complementary slackness condition if for $\beta \geq 1$ we have either $y_i > 0$ or $b_j \leq \sum_i a_{ij} x_i \leq b_j \beta$ for any $j$.
\end{itemize}

If $x$ and $y$ satisify the primal and dual complementary slackness conditions respectively then 
\[
\sum_{i=1}^n c_i x_i  \leq \alpha \beta  \sum_{j=1}^m b_j y_j
\]

Note that for $\alpha = \beta = 1$ this gives the ordinary complementary slackness theorem.
The proof follows directly from applying the primal and dual complementary slackness properties to $\sum_i c_i x_i$.
