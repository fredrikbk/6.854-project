\subsection{Preliminaries}
In addition to the standard theorems and definitions about linear programs and duality we introduce \emph{approximate} complementary slackness.
First, we review the concept of ordinary complementary slackness.
Consider the following linear programs:

\vspace{-.5cm}
\begin{align*}
& \textnormal{Primal: } \textnormal{minimize} \sum_{i=1}^n c_i x_i \\
 &\qquad \textnormal{subject to } \sum_{i=1}^n a_{ij} x_i  \geq b_j \, , \;\; x_i \geq 0 \\
& \textnormal{Dual: } \textnormal{maximize} \sum_{j=1}^m b_j y_j \\
 &\qquad \textnormal{subject to } \sum_{j=1}^m a_{ij} y_j \leq c_i \, , \;\; y_j \geq 0
\end{align*}
\vspace{-.5cm}

A feasible solution $x = (x_i, \ldots, x_n)$ to the primal problem satisifies the complementary slackness condition if and only if for each $i$ we have either $x_i = 0$ or $\sum_{j} a_{ij}y_j = c_i$.
So either the variable $x_i$ is equal to 0 or the corresponding constraint in the dual is tight.
A feasible solution $x$ is optimal if and only if it satisifies the complementary slackness condition.

Approximate slackness relaxes the tightness conditions above.
As a result, we only get an approximation ratio between the primal and dual solutions satisfying the approximate complementary slackness condition.
As we will see in the context of online algorithms, this approximation ratio corresponds to the competitive ratio.

Now we precisely state the approximate complementary slackness condition.
Let $x$ and $y$ be feasible solutions to the primal and dual linear programs.
First we define the primal and dual complementary slackness conditions.

A feasible solution $x$ satisfies the \emph{primal} complementary slackness condition if for $\alpha > 1$ we have $x_i > 0$ or $c_i / \alpha \leq \sum_j a_{ij} y_j \leq c_i$ for any $i$.
Similarly, $y$ satisifies the \emph{dual} complementary slackness condition if for $\beta > 1$ we have $y_i > 0$ or $b_j \leq \sum_i a_{ij} x_i \leq b_j \beta$.

If $x$ and $y$ satisify the primal and dual complementary slackness conditions respectively then 
\[
\sum_{i=1}^n c_i x_i  \leq \alpha \beta  \sum_{j=1}^m b_j y_j
\]

Note that for $\alpha = \beta = 1$ this gives the ordinary complementary slackness theorem.
The proof follows directly from applying the primal and dual complementary slackness properties to $\sum_i c_i x_i$.
