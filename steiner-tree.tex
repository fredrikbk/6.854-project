In this section we explore several applications of the primal-dual framework to several online network optimization problems. We begin by looking at a very general problem called the generalized connectivity problem. In this problem there is an underlying undirected graph $G = (V,E)$ where each edge has a non-negative cost $c_e$. At each time step, a new request arrives in the form $(S_i,T_i)$ where $S_i$ and $T_i$ are two non-empty disjoint subsets of $V$ and the goal is to connect some node in $S_i$ to some node in $T_i$ via a path. For every request, we need to decide which edges to buy so that we can satisfy it. Note that edges that were previously purchased can be reused for handling future requests. The goal is to minimize the total cost of purchased edges.

\subsection{Fractional Connectivity Problem}

We are interested in solving the fractional version of the connectivity problem where we are allowed to buy fractions of the edges (capacity) and we only require that a unit of flow can be sent from nodes in $S_i$ to nodes in $T_i$. In the online setting once we bought some capacity in an edge we can never decrease it but we may increase it later on. To approach this we will apply the framework that was previously described. We first formulate the LP where for each edge we keep a variable $x_e$ indicating the capacity of the edge that we own. We note that a unit flow can be sent from $S_i$ to $T_i$ if and only if every $(S_i, T_i)$ cut with respect to $x_e$ has value at least 1. Therefore, the LP we get is: 
\[
	\begin{array}{lr}
	\textrm{Primal: minimize}   & \sum_{e \in E} c_e x_e   \\
	\textrm{subject to:} & \\
	\textrm{for each day $i$, every $(S_i,T_i)$ cut $C$} & \sum_{e \in C} x_e  \ge 1  \\
			    & x_e \geq 0
	\end{array}
\]

To get an online solution for the LP we use the framework developed in the previous section.The difference is that now instead of one constraint we get an exponential number of constraints added per request. We deal with this however by finding the constraint that corresponds to the minimum $(S_i,T_i)$ cut every time and increase the primal and the dual functions as mentioned in the previous section. So in total the algorithm does the following for every request:

\begin{algorithm}
\caption{Request($S_i$,$T_i$)}
\begin{algorithmic}[1]
\WHILE{the max $S_i \rightarrow T_i$ flow is less than 1}
  \STATE Find the minimum  $(S_i,T_i)$ cut $C$
  \FORALL{edges $e \in C$}
    \STATE $x_e \leftarrow x_e (1 + 1/c_e) + 1/(|C| \cdot c_e)$ 
  \ENDFOR
  \STATE $y_{i,C} \leftarrow y_{i,C} + 1$
\ENDWHILE
\STATE Make each edge $e$ have capacity $x_e$
\end{algorithmic}
\end{algorithm}

The algorithm above has competitive ratio $O( \log \textrm{MAX-CUT} )$ and runs in polynomial time.

\subsection{Facility Location}
We now move on to develop a randomized strategy for the integral version of the connectivity problem based on the fractional solution we got above.Since this is in general very hard to do we will restrict ourselves in a very interesting subproblem, the (non-metric) facility location.

In this problem, there is a set $F$ of possible facilities that we can build along with a certain building cost $c_f$ for every $f \in F$. In this problem clients arrive one at a time and want to be connected to at least one of the built facilities. Every client $i$ has a certain cost $c_{i,f}$ for every facility $f$ that he may be connected to. So the decision we have to make whenever a client appears is whether to connect him to some previously built facility or to build a new facility and connect him to it. Again, since this is an online setting we cannot undo the fact that a facility has been built or that a client has been connected to the facility.

We now show how to reduce this problem to the connectivity problem. For every facility $f \in F$, we create a tree where the root corresponds to $f$. We connect the root to a single node with an edge of cost $c_f$. And then we connect this node to the leaves where each leaf corresponds to every client. The edges connecting the client nodes to the intermediate node have cost equal to $c_{i,f}$. Note that each tree has unbounded size since the number of clients may be arbitrarily large and that in the online setting we don't know the edge costs of a client before he appears. However we will never use a part of the tree that corresponds to some client that hasn't arrived yet.

Every time a new client appears we need to connect a node that corresponds to him in some tree to the root of that tree by buying edges. If the edge connecting the root of the tree to the intermediate node has already been bought it means that the facility has been built and that we only need to pay for connecting the client to that facility (buy the edge to the intermediate node). Thus, in the context of the connectivity problem we can see that each client is associated with two sets $(S_i,T_i)$ where $S_i$ is the set of nodes corresponding to him in every tree and $T_i$ are the roots of all trees.

Therefore we can use the fractional algorithm we developed previously to get a fractional solution to the problem. Since the cut every time is exactly equal to $m$ where $m$ is the number of trees/possible facility locations the competitive ratio is $O(\log m)$. We now see how we can use the fractional solution to obtain an online randomized algorithm for facility location.

Throughout the course of the algorithm, we maintain $2 \lceil log(n'+1) \rceil$ values for every tree when $n'$ requests have occurred so far. This values are drawn uniformly at random from $[0,1]$. Before any requests occur no tree has any such value. Whenever a request happens we sample more values so that every tree has exactly $2 \lceil log(n'+1) \rceil$. In order to decide which edges to buy we compute for every tree the minimum of its values and buy all edges with fractional capacity greater than this value. For a single value drawn randomly out of $[0,1]$ edge $e$ has probability $x_e$ of being selected. By the union bound for $2 \lceil log(n'+1) \rceil$ values the edge has probability less than $2 \lceil log(n'+1) \rceil x_e$ of being selected. So in total the expected cost is at most $2 \lceil log(n'+1) \rceil = O(\log n')$ times the cost of the fractional solution. This means that it is at most $O(\log n \log m)$ times the optimal offline algorithm.

The resulting solution however may not be feasible since it may be the case that a client is not connected to any facility.
This happens with low probability: $1/n'^{2}$~\cite{naor11:node-weighted-steiner-tree}.
We can therefore connect client $i$ to some root via the cheapest path. This is a lower bound to the value of the optimal offline solution since OPT has to connect client $i$ to some facility. So overall the additional expected cost added when the randomization fails to connect a client to a facility is $\sum_{n'} OPT/n'^2 = O(1) OPT$ which is negligible.

So the competitive ratio of the randomized algorithm is $O(\log n \log m)$ for a total of $n$ requests.

\subsection{Node-Weighted Steiner Tree}

In the node weighted Steiner tree problem, there is an underlying graph $G=(V,E)$ where both vertices and edges have costs ($c_v$ and $c_e$) respectively.At all times, we maintain a network of connected nodes in the graph. Initially this network is empty. Every time a request arrives on a vertex of the graph and it has to be connected to the network via a path formed by nodes and edges that we buy. Again we need to decide every time which nodes and edges to buy to minimize the cost of keeping all request points connected. Once we buy a node or an edge we can never undo this action.

This problem is very similar to the previous problems and in fact we will show how we can reduce it to facility location to obtain an online algorithm. To do this we use the following lemma ~\cite{naor11:node-weighted-steiner-tree}:

\begin{lemma}
For any set of nodes $S = \{s_1,...,s_k\}$ and any tree $T \subseteq G$ that contains all nodes in $S$, 
there exist nodes $v_2, v_3, ... , v_k \in T$, not necessarily distinct, for which:
\[ \sum_{i=2}^k [c(P_i) - c_{v_i}] \le O(\log k) c(T) \]
where $c(P_i)$ is the cost of the cheapest path that connects node $s_i$ to some previous node $s_j$ with $j<i$ while going through $v_i$ and $c(T)$ is the cost of the tree. Costs include both node and edge costs.
\end{lemma}

Using this lemma we observe that for the optimal Steiner tree $T^*$ and its corresponding $v_2, v_3, ... , v_k \in T^*$ it holds that:
\[ \sum_{i=2}^k [c(P_i) - c_{v_i}] + \sum_{v \in \{v_2, v_3, ... , v_k\}} c_{v} \le O(\log k) c(T^*) \]
where the second summation is over the distinct nodes $v_2, v_3, ... , v_k$ and it is at most $c(T^*)$.

It follows that the quantity that minimizes $\sum_{i=2}^k [c(P_i) - c_{v_i}] + \sum_{v \in \{v_2, v_3, ... , v_k\}} c_{v}$ for some nodes $v_2,...,v_k$ is $O(\log k)$ approximate for the node weighted Steiner tree problem. This problem can be viewed as a facility location problem where nodes in the graph are the candidate facility locations and the chosen $v_i$ correspond to the built facilities. Every time a request comes at a node $v$ we calculate the minimum cost of connecting it to some previously requested node for every node $i$ in the graph without including the cost for node $i$. Viewed as a facility location problem this cost corresponds to the cost of connecting the new client to facility $i$. Using the previously developed algorithm for facility location we get a $O(\log^2 k \log n)$ where $k$ is the number of requests and $n$ is the number of nodes in the graph. 

