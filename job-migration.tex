In this section we demonstrate the applicability of the primal-dual method to real-world problems.
We survey a systems paper, which is primarily interesting because of its applicability to systems design in the real-world, and not necessarily because of its theoretical value.

Linear programs are able to capture many real world problems, and the primal-dual technique gives us a tool that lets us to design online algorithms from many linear programs.
We we illustrate this here by discussing its applicability to a data center job-migration problem.

The goal of the migration is to move jobs to data centers at locations with temporarily cheaper power.
This algorithm was developed by Buchbinder, Jain and Menache in their 2011 paper \emph{``Online Job-Migration for Reducing the Electricity Bill in the Cloud''} and the companion technical report with the same name~\cite{buchbinder11:job-migration,buchbinder11:job-migration-techreport}.
The primal-dual algorithm developed by the authors using the primal-dual method turns out to be too computationally expensive to be used in practice.
However, it serves as the inspiration for a non-intuitive algorithm that they empirically show to outperform two reasonable greedy heuristics and to be within $4\%-6\%$ of the optimal, on real-world data.

\subsection{Problem Formulation}

% Explain/Motivate problem
Energy is quickly becoming the dominant cost factor for data center operation.
It is expected that by 2014, the infrastructure and energy cost of data centers will be 75\% of the total cost.
The EPA reported the energy cost of server and data center operation in 2007 to be \$4.5 billion, with demand expected to double by 2011.
% Citation Needed
Furthermore, CNN reports data centers to account for 2\% of all electricity consumed in the US.
With such high stakes, it is no surprise that techniques to reduce energy costs in data centers are in high demand.

One technique that has recently been proposed to reduce energy costs when operating multiple geographically distributed data centers is to move tasks to data centers with cheaper energy.
% Citation Needed
This approach is motivated by the observation that energy prices are localized and that computation will therefore be cheaper at some sites.
Furthermore, energy prices are fluctuating and the cheapest data center at one point in time, may not be the cheapest later on.
This motivates online algorithms that monitor electricity prices and at any time move tasks to data centers where the electricity is the cheapest.
However, moving data is not free and a cost proportional to the amount of data used is typically incurred.
These competing constraints make the problem non-trivial, and greedy heuristics with provable performance are not feasible.

% Describe the model and the variables
Specifically, our problem consist of $n$ data centers, where each data center $i$ contains $H_i$ servers.
The job-load in the system is given by the constant variable $B$ (in the technical report the authors also generalizes to variable workloads).
At each discretized time step $t$ we must decide whether to move jobs between data centers, taking the bandwidth cost and energy costs into account.
The migration cost for moving data out of data center $i$ is defined as $d_i$, and the variable $z_{i,j,t}$ is the number of jobs migrated from server $j$ in data center $i$ at time $t$.
Finally, $c_{i,j,t}$ is the energy cost of operating server $j$ in data center $i$ at time $t$, and $y_{i,j,t}$ is the fraction of server $j$ of data center $i$ that is utilized at time $t$.
Jobs are defined as using exactly one server, so the number of servers needed at data center $i$ at time $t$ is $\lceil \sum^{H_i}_{j=1}y_{i,j,t} \rceil$.
Since the number of servers in each data center is expected to be large we can ignore the extra fraction introduced by the ceiling operator.

% Define the LP
The following (primal) linear program captures the online data center job-migration problem:
\[
\textrm{(P) : min}  \sum^n_{i=1}\sum^{H_i}_{j=1}\sum^{T}_{t=1}d_{i} \cdot z_{i,j,t} + \sum^{n}_{i=1}\sum^{H_{i}}_{j=1}\sum^{T}_{t=1} c_{i,j,t} \cdot y_{i,j,t}
\]
\[
	\begin{array}{rc}
	\textrm{subject to :} & \\
		\forall t \textrm{ :} & \sum^n_{i=1}\sum^{H_i}_{j=1}y_{i,j,t} \geq B \\
		\forall i,j,t \textrm{ :}	    & z_{i,j,t} \geq y_{i,j,t-1} - y_{i,j,t} \\
		\forall i,j,t \textrm{ :}	    & y_{i,j,t} \leq 1 \\
		\forall i,j,t \textrm{ :}	    & y_{i,j,t} \geq 0 \\
		\forall i,j,t \textrm{ :}	    & z_{i,j,y} \geq 0
	\end{array}
\]

The objective function minimizes the sum of the bandwidth cost (first term) and the energy cost (second term).
The first constraint ensures the allocation satisfies the workload $B$.
The second constraint enforces that the amount of work leaving a server is at least the same as the work that server lost (it may be more as a server may also receive work).
The third prevents servers from being over-used as each server can only execute one task, and the fourth and fifth constraints just prevents server usage or bandwidth from being negative.

The corresponding dual linear program is:

\[
\textrm{(D) : max}   \sum^{T}_{t=1}B \cdot a_t - \sum^{n}_{i=1}\sum^{H_i}_{j=1}\sum^{T}_{t=1}s_{i,j,t}
\]
\[
	\begin{array}{rc}
	\textrm{s.t. :} & \\
		\forall i,j,t \textrm{ :}	    & -c_{i,j,t} + a_t + b_{i,j,t} - b_{i,j,t+1} - s_{i,j,t} \leq 0 \\
		\forall i,j,t \textrm{ :}	    & b_{i,j,t} \leq d_i \\
		\forall i,j,t \textrm{ :}	    & b_{i,j,t} \geq 0 \\
		\forall i,j,t \textrm{ :}	    & s_{i,j,t} \geq 0
	\end{array}
\]

%\subsection{Primal-Dual Online Algorithm}

% Algorithm

% Proof sketch (maybe)

%\subsection{Efficient Online Algorithm}

% Efficient Algorithm

% Experiments