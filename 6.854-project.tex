\title{A Survey of Recent Work on Designing Competitive Online Algorithms via the Primal--Dual Approach}
\author{Fredrik Kjolstad \and Ludwig Schmidt \and Christos Tzamos}
\date{}

\documentclass[10pt, twocolumn]{article}

\usepackage{amsmath}
\usepackage{amsthm}
\usepackage{url}
\usepackage{relsize}
\usepackage{xspace}
\usepackage{subfigure}
\usepackage{graphicx,color}

\usepackage{fullpage}

% Todos
\newcommand{\fred}[1]{%
	\textcolor{red}{[#1]}
}
\newcommand{\christos}[1]{%
	\textcolor{red}{[#1]}
}
\newcommand{\ludwig}[1]{%
	\textcolor{red}{[#1]}
}



\begin{document}
\maketitle

\begin{abstract}
Online algorithms have become increasingly popular in the last few decades.
An important reason for this is that they capture the uncertainty we face in many important domains, ranging from computational finance to internet security and catastrophe management.
The primal-dual method is an approach that has been gaining popularity as a technique to arrive at approximations for NP-hard problems.
More recently it has also been applied as a general framework to solve many online algorithms.
In 2009, Buchbinder and Joseph published a survey of applications of the primal-dual method to online algorithms.
Since then new complex online problems have been tackled using this technique, such as the online node-weighted steiner tree problem, the k-server problem and online job-migration.
We present a survey of recent applications of the primal-dual method to online problems.
\end{abstract}

\section{Introduction}
% Short paragraph to tie intro to abstract
The primal-dual technique was successfully applied to approximate NP-Hard problems in the 1990's.
% Citations needed
In the early 2000's an application of this technique to online algorithms was discovered, and has since then been successfully used to solve many online algorithms.
% Citations needed

% Talk about how it allows you to specify online programs as offline linear programs that are then solved incrementally online using the technique.
% Solve program incrementally by successively adding in variables and constraints
The primal-dual method allows us to continuously approximate the solution to linear programs where constraints are revealed incrementally one or more at a time. 
Furthermore, at each step we can only add to the solution and not remove values we assigned to variables at previous steps.
It is easy to see how the method maps naturally to online problems where information is revealed in pieces, and where we at each step must make irrevocable decissions based on the limited knowledge we have thus far.
In fact, the primal-dual method provides us with a very general framework that lets us turn linear programs solving \emph{offline} problems into approximate solvers for their \emph{online} counter-parts.

% Buchbinder and Joseph's work
In 2009 Buchbinder and Naor published a survey of the application of the primal-dual method to online algorithms~\cite{buchbinder09:survey}.
The survey gave a thorough description of the technique, and went on to survey several papers showing how the primal-dual method can be applied to solve a number of important online problems. 
Examples include online set-cover, caching, routing and ad-auction revenue maximization.

Since then the primal-dual method has been applied to solve additional problems in an online setting.
In this paper we survey three new application of the technique to very different problems.
This demonstrates the power and applicability of the primal-dual approach to online problems.

In section~\ref{primal-dual} we offer an introduction to the primal-dual method using ski rental as a running example.
In section~\ref{steiner-tree} we survey Naor, Panigrahi and Singh's paper from 2011, which presents the first online algorithm for the node-weighted steiner tree problem~\cite{naor11:node-weighted-steiner-tree}.
\fred{Christos: Could you add one sentence on your paper here?}
In section~\ref{job-migration} we survey Buchbinder, Jain and Menache's 2011 paper and companion technical report, in which they design an algorithm for online job-migration between geographically distributed cloud data centers to reduce electricity costs~\cite{buchbinder11:job-migration,buchbinder11:job-migration-techreport}.
This paper demonstrates the flexibility of the primal-dual technique, by applying it to the design of algorithms for complex real-world problems that beats reasonable greedy heuristics.
Finally, section~\ref{k-server} surveys Bansal, Buchbinder and Naor's 2010 paper, which demonstrates the power of the primal-dual approach by using it to explore promising approaches to the online k-server problem: the holy grail of online algorithms~\cite{bansal10:k-server}.




\section{The Primal-Dual Approach}

In this section we present the primal-dual approach.
First, we explain the required background in linear programming and duality.
We then introduce the technique using the well known ski rental problem as an example.
Finally, we discuss the general primal--dual framework so that it can be applied to other problems.

We assume the reader is already familiar with the online algorithm model, competitive analysis, approximation algorithms, (integer) linear programming, randomized rounding and the concepts of weak and strong duality and will not review these concepts here.
For a brief discussion of these concepts we refer the reader to section~2 of Buchbinder and Naor's survey~\cite{buchbinder09:survey}.

\subsection{Preliminaries}
In addition to the standard theorems and definitions about linear programs and duality we introduce \emph{approximate} complementary slackness.
First, we review the concept of ordinary complementary slackness.
Consider the following linear programs:

\vspace{-.5cm}
\begin{align*}
& \textnormal{Primal: } \textnormal{minimize} \sum_{i=1}^n c_i x_i \\
 &\qquad \textnormal{subject to } \sum_{i=1}^n a_{ij} x_i  \geq b_j \, , \;\; x_i \geq 0 \\
& \textnormal{Dual: } \textnormal{maximize} \sum_{j=1}^m b_j y_j \\
 &\qquad \textnormal{subject to } \sum_{j=1}^m a_{ij} y_j \leq c_i \, , \;\; y_j \geq 0
\end{align*}
\vspace{-.5cm}

A feasible solution $x = (x_i, \ldots, x_n)$ to the primal problem satisifies the complementary slackness condition if and only if for each $i$ we have either $x_i = 0$ or $\sum_{j} a_{ij}y_j = c_i$.
So either the variable $x_i$ is equal to 0 or the corresponding constraint in the dual is tight.
A feasible solution $x$ is optimal if and only if it satisifies the complementary slackness condition.

Approximate slackness relaxes the tightness conditions above.
As a result, we only get an approximation ratio between the primal and dual solutions satisfying the approximate complementary slackness condition.
As we will see in the context of online algorithms, this approximation ratio corresponds to the competitive ratio.

Now we precisely state the approximate complementary slackness condition.
Let $x$ and $y$ be feasible solutions to the primal and dual linear programs.
First we define the primal and dual complementary slackness conditions.

A feasible solution $x$ satisfies the \emph{primal} complementary slackness condition if for $\alpha > 1$ we have $x_i > 0$ or $c_i / \alpha \leq \sum_j a_{ij} y_j \leq c_i$ for any $i$.
Similarly, $y$ satisifies the \emph{dual} complementary slackness condition if for $\beta > 1$ we have $y_i > 0$ or $b_j \leq \sum_i a_{ij} x_i \leq b_j \beta$.

If $x$ and $y$ satisify the primal and dual complementary slackness conditions respectively then 
\[
\sum_{i=1}^n c_i x_i  \leq \alpha \beta  \sum_{j=1}^m b_j y_j
\]

Note that for $\alpha = \beta = 1$ this gives the ordinary complementary slackness theorem.
The proof follows directly from applying the primal and dual complementary slackness properties to $\sum_i c_i x_i$.

\subsection{Ski Rental}

In this section we informally present the primal-dual approach by applying it to the well-known ski rental problem.

In the ski rental problem a skier will go skiing several times in his life.
Every time he goes skiing he has to decide whether to rent a pair of skis or to buy a pair of skis that he can use for all subsequent ski trips.
Renting costs \$1, while buying skis costs \$B.
The goal of the skier is to spend the least amount of money.
The ski rental problem is interesting because the skier does not know beforehand how many days he will ski in his life --- after all he may break his leg tomorrow.
We assume an adverserial model where fate will ensure the worst possible outcome no matter what the skier decides.

\subsubsection{The Online Ski Rental Algorithm}
Against a malicious adversary, the optimal deterministic strategy that minimizes the competitive ratio has been known for a long time, and requires the skier to rent for $B$ days before buying.
If the last day of skiing is before day B then this strategy is optimal.
On the other hand, if the last day of skiing is after day B, then the strategy is at most two times as expensive as the optimal strategy, OPT, which is to buy skis on the first day of skiing.
This strategy achieves a competitive ratio of 2 (it is 2-competitive), and is optimal for deterministic strategies.

However, a better competitive ratio of $\frac{e}{e-1}$ can be achieved using randomization.
We will provide an optimal randomized algorithm using the primal-dual approach, but first we will discuss a deterministic primal-dual algorithm that achieves 2-competitiveness.

\subsubsection{LP Formulation}
We begin by formulating an integer linear program that captures the solution for the offline ski rental problem.
We use an indicator variable $x \in \{0,1\}$ that represents whether we buy the skis and a variable $d_i \in \{0,1\}$ for every day that indicates whether we rent skis at day $i$. 
The objective we want to minimize then is:
\[ B\cdot x + \sum^n_{i=1} d_i \]
subject to the constraints that for each day $i$:
\[ x + d_i \ge 1 \]

The optimal solution in an offline version is either $x=1$ and $d_i = 0$ for each day or $x=0$ and $d_i = 1$ for each day, whichever is best.
That is, the optimal solution is to buy at once or never buy at all.
This of course requires prior knowledge of $n$ which we do not have in an online setting.

In an online setting we know the objective beforehand, but we begin without any constraints.
Every day, a new constraint appears and we have to re-optimize our solution.
However, since we cannot change the past we cannot undo any previous decisions.
This means that we can never decrease any variables set in the past.

\subsubsection{Algorithm}

We will now describe how we can incrementally solve the linear program using the primal-dual approach.
We will later argue that this will not cause the solution to deviate too much from OPT.
In order to apply the primal-dual approach, we first have to relax the integer linear program to a linear program. Note that the optimal solution in the relaxed version is exactly the same as before.

The primal-dual method approximates the solution to the linear program incrementally by updating the primal and dual simultaneously.
The primal and dual is given below.

\[
	\begin{array}{lr}
	\textrm{Primal: minimize}   & B\cdot x + \sum^n_{i=1} d_i   \\
	\textrm{subject to:} & \\
	\textrm{for each day $i$} & x + d_i  \ge 1  \\
			    & x     \geq 0, \forall i : d_i \ge 0
	\end{array}
\]
\[
	\begin{array}{lr}
	\textrm{Dual: maximize}   & \sum^n_{i=1} y_i   \\
	\textrm{subject to:} & \sum^n_{i=1} y_i \le B \\
	\textrm{for each day $i$} & 0 \le y_i  \le 1
	\end{array}
\]

For the online version, in the $i$-th day a new constraint ($x + d_i  \ge 1$) arrives in the primal problem and a new variable ($y_i$) arrives in the dual. 
With the new constraint the primal problem may become infeasible (this happens if $x < 1$). 
If it is still feasible we do not have to do anything since we already covered the new constraint. 
Otherwise, we need to decide which variable, $x$ or $d_i$, to increase. 
To make this decision we look at the dual. 
Since the dual is a maximization problem, we would like to increase the new variable $y_i$ as much as possible.
Thus we increase $y_i$ until we hit a constraint in the dual problem.
This constraint corresponds to a variable in the primal problem, either $d_i$ or $x$.
We set this variable equal to 1.

Notice that the algorithm corresponds to exactly the same strategy we described previously, i.e. the skier rents skis for the first B days and then buys them.

\subsubsection{Analysis}

The analysis of the ski rental problem is crucial for understanding the concepts of the primal-dual method. 
The key idea is that every time the algorithm makes a change in the primal LP, a change in the dual LP happens as well so that the ratio between their corresponding objective values remains bounded by some function. 
This function gives the competitive ratio of the algorithm.

For the previous algorithm, we note that the update rule always maintains a feasible solution for both the primal and the dual. 
In order to bound the ratio between their objectives we use approximate complementary slackness. 
By the update rule we have that:
\begin{itemize}
\item Whenever $x > 0$, $\sum^n_{i=1} y_i = B$ (tight)
\item Whenever $d_i > 0$, $y_i = 1$ (tight)
\item Whenever $y_i > 0$, $1 \le x+d_i \le 2$ (approximately tight)
\end{itemize}

So by approximate complementary slackness with $\alpha = 1$ and $\beta = 2$ we get that the primal objective is at most 2 times the dual objective. Therefore, it is at most 2 times the optimal offline solution and thus it is 2-competitive.

\subsubsection{Fractional Algorithm Analysis}
We now move on to get a randomized algorithm for the ski rental problem. 
We first try to find a fractional online solution and then use the fractional values of the primal variables to get the probabilities of buying and renting the skiis for each day.
In the deterministic case we always maintained a feasible primal solution and whenever a new constraint appeared on day $i$ we either set $d_i$ or $x$ to 1. For the fractional case we do not have to set $x$ directly to 1 when we reach $B$ days but we can gradually increase $x$ every day. So for every new day $i$ if $x<1$ we use the update rule:
\begin{itemize}
\item $x \leftarrow x(1+1/B) + 1/(c\cdot B)$
\item $d_i \leftarrow 1 - x$
\item $y_i \leftarrow 1$
\end{itemize}
where $c$ is a constant to be defined later.

We move on to the analysis part of the algorithm. Let $x^{(i)}$ be the value of $x$ on the $i$-th day. We first note that under the update rule chosen, after $k$ days $x^{(k)} =  \sum_{i=0}^{k-1} (1+1/B)^i / (c\cdot B) = [(1+1/B)^k-1]/c$. It is obvious that the primal solution is always feasible since at any time $i$, $x^{(i)}+d_i=1$ and no previous constraints are violated by the increase in $x$. We choose $c$ so that the dual is feasible as well. We want $y_i$ to be 1 for at most $B$ days otherwise the dual constraint  $\sum^n_{i=1} y_i \le B$ will be violated. So we want that after $B$ days $x = 1$ which means that $c = (1+1/B)^k-1$.

Moreover, we want the ratio between the primal objective and the dual objective to be bounded by a constant.
We use a proof by induction to show that it is bounded by $(1+1/c)$.
Initially, the objective functions of the primal and the dual are both 0 so the statement holds.
Assuming that it holds for $k$ days, we now prove that it holds for $k+1$ days.
If $k\ge B$ it clearly holds since the two objectives are left unchanged.
Otherwise, we see that x increases by $x^{(k)} / B + 1/(c \cdot B)$ while $d_{k+1}$ becomes $1 - x^{(k+1)}$.
So the primal objective increases by $x^{(k)} + 1/c + 1 - x^{(k+1)} \le 1 + 1/c$ since $x^{(k)} \le x^{(k+1)}$.
On the other hand the dual objective increases by 1 unit so the statement holds.

Therefore we get an online fractional algorithm for the ski rental problem with competitive ratio $1+1/c \approx e/(e-1)$ for large values of $B$.
Following we see how to make this randomized based on the fractional values we get.

\subsubsection{Randomized Ski Rental}
Our randomized strategy for deciding is as follows: In the beginning we pick a random value $v$ in the interval $[0,1]$ and we buy skis on the first day that $x \ge v$.
The probability of buying skis in the first $n$ days is exactly $x^{(n)}$, i.e. the assigned fractional value by the online fractional solution. The probability of renting skis in the $i$-th day is exactly $1-x^{(i)}$.
If we go skiing for $n$ days the expected cost of our randomized algorithm is therefore $B \cdot  x^{(n)} + \sum_{i=1}^n(1-x^{(i)})$ which is exactly the same as the objective of the fractional algorithm.
Thus the randomized algorithm is also $e/(e-1)$-competitve.



\section{Online Node-weighted Steiner Tree}
In this section we explore several applications of the primal-dual framework to three online graph optimization problems. We examine fractional connectivity, non-metric facility location (TODO cite) and node-weighted Steiner tree (TODO cite). The best known algorithms for the metric facility location and the Steiner-tree problems have $O(\frac{ \log n }{ \log \log n} )$ (TODO cite) and $O({ \log n} )$(TODO cite) competitive ratios respectively. We see how the framework allows us to easily obtain compareble results in generalizations of those problems where no results were previously known.

\subsection{Fractional Connectivity}

In the fractional connectivity problem there is an underlying undirected graph $G = (V,E)$ where each edge $e$ has a non-negative cost $c_e$. At each time step, a new request arrives in the form $(S_i,T_i)$ where $S_i$ and $T_i$ are two non-empty disjoint subsets of $V$ and the goal is to send a unit of flow from $S_i$ to $T_i$. To do this we buy capacity on the edges. If we buy capacity $x_e$ on an edge $e$ we have to pay $x_e \cdot c_e$. In the online setting once we buy capacity in an edge we can reuse it for all subsequent requests. Also, we can never decrease the capacity but we may increase it later on by paying the extra cost.

We now try to apply the framework to this problem. We first formulate the LP where for each edge we keep a variable $x_e$ indicating the capacity of the edge that we own. We note that a unit flow can be sent from $S_i$ to $T_i$ if and only if every $(S_i, T_i)$ cut with respect to $x_e$ has value at least 1. Therefore, the LP we get is: 
\[
	\begin{array}{lr}
	\textrm{Primal: minimize}   & \sum_{e \in E} c_e x_e   \\
	\textrm{subject to:} & \\
	\textrm{for each day $i$, every $(S_i,T_i)$ cut $C$} & \sum_{e \in C} x_e  \ge 1  \\
			    & x_e \geq 0
	\end{array}
\]

To get an online solution for the LP we use the framework developed in the previous section.The difference is that now instead of one constraint we get an exponential number of constraints added per request. We deal with this however by finding the constraint that corresponds to the minimum $(S_i,T_i)$ cut every time and increase the primal and the dual variables as mentioned in the previous section. So in total the adjusted algorithm \ref{generalalg} does the following for every request:

\begin{algorithm}
\caption{Request($S_i$,$T_i$)}
\begin{algorithmic}[1]
\WHILE{the max $S_i \rightarrow T_i$ flow is less than 1}
  \STATE Find the minimum  $(S_i,T_i)$ cut $C$
  \FORALL{edges $e \in C$}
    \STATE $x_e \leftarrow x_e (1 + 1/c_e) + 1/(|C| \cdot c_e)$ 
  \ENDFOR
  \STATE $y_{i,C} \leftarrow y_{i,C} + 1$
\ENDWHILE
\STATE Make each edge $e$ have capacity $x_e$
\end{algorithmic}
\end{algorithm}

As mentioned in the previous section, the algorithm is $O(\log d)$-competitive, where $d = \max_j\{|S_j|\}$. Here each $S_j$ corresponds to a cut in the graph so $O( \log d ) = O( \log \textrm{MAX-CUT} ) = O( \log n )$ where $n = |V|$, the number of nodes in the graph.

The integral version of the connectivity problem where we are only allowed to buy the whole edge and not fractions of it is of particular interest since it captures many graph optimization problems. By randomly rounding the fractional capacity values in every edge we can obtain an integral solution. The facility location as we will see next is one example for this.

\subsection{Facility Location}
In the facility location problem, there is a set $F$ of $n$ possible facilities that we can build along with a certain building cost $c_f$ for every $f \in F$. In this problem clients arrive one at a time and want to be connected to at least one of the built facilities. Every client $i$ has a certain cost $c_{i,f}$ for every facility $f$ that he may be connected to. So the decision we have to make whenever a client appears is whether to connect him to some previously built facility or to build a new facility and connect him to it. Again, since this is an online setting we cannot undo the fact that a facility has been built or that a client has been connected to the facility.

We now show how to reduce this problem to the integral connectivity problem. For every facility $f \in F$, we create a tree where the root corresponds to $f$. We connect the root to a single node with an edge of cost $c_f$. And then we connect this node to the leaves where each leaf corresponds to every client. The edges connecting the client nodes to the intermediate node have cost equal to $c_{i,f}$. Note that each tree has unbounded size since the number of clients may be arbitrarily large and that in the online setting we don't know the edge costs of a client before he appears. However we will never use a part of the tree that corresponds to some client that hasn't arrived yet.

Every time a new client appears we need to connect a node that corresponds to him in some tree to the root of that tree by buying edges. If the edge connecting the root of the tree to the intermediate node has already been bought it means that the facility has been built and that we only need to pay for connecting the client to that facility (buy the edge to the intermediate node). Thus, in the context of the connectivity problem we can see that each client is associated with two sets $(S_i,T_i)$ where $S_i$ is the set of nodes corresponding to him in every tree and $T_i$ are the roots of all trees.

Therefore we can use the fractional algorithm we developed previously to get a fractional solution to the problem. Since the cut every time is exactly equal to $n$ where $n$ is the number of trees/possible facility locations the competitive ratio is $O(\log n)$. We now see how we can use the fractional solution to obtain an online randomized algorithm for facility location.

Throughout the course of the algorithm, we maintain $2 \lceil log(k+1) \rceil$ values for every tree when $k$ requests have occurred so far. This values are drawn uniformly at random from $[0,1]$. Before any requests occur no tree has any such value. Whenever a request happens we sample more values so that every tree has exactly $2 \lceil log(k+1) \rceil$. In order to decide which edges to buy we compute for every tree the minimum of its values and buy all edges with fractional capacity greater than this value. For a single value drawn randomly out of $[0,1]$ edge $e$ has probability $x_e$ of being selected. By the union bound for $2 \lceil log(k+1) \rceil$ values the edge has probability less than $2 \lceil log(k+1) \rceil x_e$ of being selected. So in total the expected cost is at most $2 \lceil log(k+1) \rceil = O(\log k)$ times the cost of the fractional solution. This means that it is at most $O(\log n \log k)$ times the optimal offline algorithm.

The resulting solution however may not be feasible since it may be the case that a client is not connected to any facility.
This happens with low probability: $1/k^{2}$~\cite{naor11:node-weighted-steiner-tree}.
We can therefore connect client $i$ to some root via the cheapest path. This is a lower bound to the value of the optimal offline solution since OPT has to connect client $i$ to some facility. So overall the additional expected cost added when the randomization fails to connect a client to a facility is $\sum_{k} OPT/k^2 = O(1) OPT$ which is negligible.

So the competitive ratio of the randomized algorithm is $O(\log n \log k)$ for a total of $k$ requests.

\subsection{Node-Weighted Steiner Tree}

In the node weighted Steiner tree problem, there is an underlying graph $G=(V,E)$ where both vertices and edges have costs ($c_v$ and $c_e$) respectively. At all times, we maintain a network of connected nodes in the graph. Initially this network is empty. Every time a request arrives on a vertex of the graph and it has to be connected to the network via a path formed by nodes and edges that we buy. Again we need to decide every time which nodes and edges to buy to minimize the cost of keeping all request points connected. Once we buy a node or an edge we can never undo this action.

To obtain an online algorithm for this problem we use the following lemma ~\cite{naor11:node-weighted-steiner-tree}:

\begin{lemma}
For any set of nodes $S = \{s_1,...,s_k\}$ and any tree $T \subseteq G$ that contains all nodes in $S$, 
there exist nodes $v_2, v_3, ... , v_k \in T$, not necessarily distinct, for which:
\[ \sum_{i=2}^k [c(P^{v_i}_i) - c_{v_i}] + \sum_{v \in \{v_2, v_3, ... , v_k\}} c_{v} \le O(\log k) c(T) \]
where $c(P^{v_i}_i)$ is the cost of the cheapest path that goes through $v_i$ and connects node $s_i$ to some previous node $s_j$ with $j<i$ and $c(T)$ is the cost of the tree. Costs include both node and edge costs.
\end{lemma}

Intuitively this lemma says that if we are careful not to double-count the cost of some nodes while possibly counting multiple times the cost of the edges and all other nodes, we don't lose more than a logarithmic factor.

It follows that minimizing the quantity $\sum_{i=2}^k [c(P_i) - c_{v_i}] + \sum_{v \in \{v_2, v_3, ... , v_k\}} c_{v}$ for some nodes $v_2,...,v_k$ we get an $O(\log k)$ approximation for the node weighted Steiner tree problem. To minimize the quantity in an online way, we can formulate the problem as a fractional connectivity problem and then try to round the fractional values. However since this is very similar to the facility location, we will formulate the problem as a non-metric facility location instance.

The facility set $F$ is the same as the set $V$ of nodes and the corresponding costs are $c_f = c_v$. We view every request $i$ at some node $s_i$ as a client wanting to be connected to some facility where the cost for each facility $f$ is $c_{i,f} = c(P^{f}_i) - c_{f}$ as in the lemma. Connecting the client $i$ to a facility $f$ corresponds to connecting the node $s_i$ to some previous request node $s_j$ with $j<i$ using the shortest path through node $f$. Using the previously developed algorithm for non-metric facility location and losing the additional $O(\log k)$ factor by the lemma, we get an $O(\log^2 k \log n)$-competitive algorithm for the node-weighted Steiner tree ($k$ is the number of requests and $n$ is the number of nodes in the graph). 

\section{Online Job-Migration}
~\cite{buchbinder11:job-migration} ~\cite{buchbinder11:job-migration-techreport}

\section{Towards the randomized k-server conjecture: a primal-dual approach}
Recently, the primal-dual technique has been used to explore new approaches to the $k$-server problem~\cite{bansal10:k-server}.
Towards this end, the authors extend the primal-dual technique to deal with a wider range of LPs.
We first demonstrate the extended primal-dual technique with an example from~\cite{bansal10:k-server}.
Then we briefly give an overview of the steps involved in applying the primal-dual technique to the $k$-server problem.

\subsection{Extended Primal-Dual Technique}
One of the main limitations of the original primal-dual technique is the restriction to non-negative constraints and variables.
Using the basic approach described in section \ref{section:general_approach}, it is impossible to include constraints like $x_i \geq x_j$, which have to be converted to $x_i - x_j \geq 0$ in an LP (note the negative coefficient of $x_j$).
This condition restricts the set of problems we can model and makes some LP formulations more complex.
The main technical contribution of~\cite{bansal10:k-server} is to overcome these limitations.

We illustrate the extended primal-dual framework with the weighted caching problem.
In this problem, we have a cache of size $k$ and a total of $n$ pages, each with a weight $w_i$.
At each time step $t$, we receive a request for a page $p_t$.
If $p_t$ is not in the cache, we have to evict one page from the cache and replace it with $p_t$.
Fetching a page $p$ into the cache costs $w_p$.
The goal is to minimize the total cost of fetching pages.

We now present a randomized $O(\log k)$ competitive algorithm for the weighted caching problem.
While a randomized $O(\log k)$ competitive algorithm was known before~\cite{bansal10:k-server}, the following algorithm using the extended primal-dual technique has a significantly simpler analysis.
It is worth noting that the first randomized $O(\log k)$ competitive algorithm also uses the primal-dual technique~\cite{bansal07:weighted-paging}.

First, we define the primal LP.
\[
\textrm{($P$) : min}  \sum_{p=1}^n\sum_{t=1}^T w_p \cdot z_{p,t} + \sum_{t=1}^T \infty \cdot y_{p_t,t}
\]
\[
	\begin{array}{rr}
	\textrm{subject to :} & \\
		\forall t \textrm{ and } S \subseteq [n] \textrm{ with }|S| > k \textrm{ :} & \sum_{p\in S} y_{p,t} \geq |S| - k \\
		\forall t,p \textrm{ :} & z_{p,t} \geq y_{p,t-1} - y_{p,t} \\
		\forall t,p \textrm{ :} & z_{p,t}, y_{p,t} \geq 0 \\
	\end{array}
\]

The variable $y_{p,t}$ denotes the fraction of page $p$ missing at time $t$.
Since $z_{p,t} \geq y_{p,t-1} - y_{p,t}$, the variable $z_{p,t}$ denotes the fraction of page $p$ fetched at time $t$.
Hence the first term of the objective captures the total cost of fetching pages.
The second term of the objective guarantees that the requested page is always fetched into the cache.
The first constraint ensures that only a total of $k$ pages are in the cache at any time.

The dual LP is
\[
\textrm{($D$) : max}  \sum_{t=1}^T\sum_S (|S| - k) a_{S,t}
\]
\[
	\begin{array}{rr}
	\textrm{subject to :} & \\
	\forall t,p \neq p_t \textrm{ :} & \sum_{S: p \in S} a_{S,t} - b_{p, t+1} + b_{p,t} \leq 0 \\
	\forall t,p \textrm{ :} & b_{p,t} \leq w_p \\
	\forall t,p \textrm{ and } |S| > k \textrm{ :} & a_{t,S}, b_{p,t} \geq 0 \\
	\end{array}
\]

We now sketch the corresponding algorithm for iteratively constructing a solution to the primal.
The following relation between the primal and dual is maintained at all times:
\[
y_{p,t} \leftarrow \frac{1}{k} \cdot \left( \exp \left( \frac{b_{p,t+1}}{w_p} \cdot \ln(1+k) \right) - 1 \right)
\]
When a new request for page $p_t$ arrives, we set all variables $y_{p,t}$ for $p \neq p_t$ to the previous value for time $t-1$.
Moreover, we set $y_{p_t, t} = b_{p_t, t} = 0$ because page $p_t$ has to be in the cache.
This violates the first primal constraint.
In order to evict pages, we now increase $b_{p, t+1}$ for all pages that are (partially) in the cache.
Note that increasing $b_{p, t+1}$ decreases the corresponding $y_{p,t}$.
Moreover, we increase $a_{S,t}$ at the same rate so that the first dual constraint remains satisfied.
Here, $S$ is the set of pages that are at least partially in the cache.

By construction, the algorithm above always maintains feasible primal and dual solutions.
Moreover, we can show that the increase in the primal objective is at most $O(\log k)$ times the change in dual objective.
This is done by considering the derivatives of the primal and dual objectives with respect to $a_{S,t}$ and $b_{p,t+1}$.
Since the relative change in primal and dual is bounded by $O(\log k)$, this shows that the fractional solution is $O(\log k)$ competitive.
Moreover, the fractional solution can be rounded to a randomized algorithm with constant overhead in the competitive ratio~\cite{bansal07:weighted-paging}.


\subsection{The $k$-Server Problem}
The $k$-server problem is one of the most important problems in the field of online algorithms.
The problem statement is as follows:
We have $k$ servers located at up to $k$ points in a metric space (essentially a set with a well-behaved distance function) and must serve requests appearing one after another.
Each request is a point in the metric space and we serve it by moving one of the $k$ servers to the request.
The goal is to minimizhee the total distance travelled by all servers.

It is conjectured that there is a deterministic $k$-competitive algorithm for the $k$-server problem.
While the best algorithm for the general case is $2k -1$ competitive, there are $k$-competitive algorithms for some special metrics like trees.
For the randomized case, the conjectured competitive ratio is $O(\log k)$ and there has been recent work showing an $\tilde{O}(log^3 n \; log^2 k)$ approximation ratio~\cite{bansal11:randomized-k-server}.
While the results in \cite{bansal11:randomized-k-server} do not use the primal-dual approach for online algorithms, some of the techniques are inspired by problems that were first introduced in the context of the primal-dual approach.

One key difficulty of the $k$-server problem on general metric spaces is the weak structure of the distances between points.
The most promising approach for overcoming this obstacle is to embed general metrics into hierarchically separated trees (HSTs).
The distortion resulting from this embedding would still be sufficient to provide a polylogarithmic competitive ratio for the randomized $k$-server problem.
Moreover, HSTs allow the $k$-server problem to be split recursively into subproblems at each node.
The advantage of this approach is that it leads to simpler subproblems since all children of a node in an HST have the same distance (a \emph{uniform} metric space).

In~\cite{bansal10:k-server}, the authors solve a restricted version of this subproblem called the \emph{Allocation-C problem}.
It is defined as follows:
We have a uniform metric space with $n$ points and up to $k$ servers.
For each request $t$, the number of available servers is limited to $k(t) \leq k$.
Each request is described by a cost vector $h^t = (h^t(0), \ldots, h^t(k))$ where $h^t(j)$ is the cost of serving the request $t$ with $j$ servers.
The cost vectors satisfy two properties:
\begin{itemize}
\item The costs are monotonic, i.e.\ $h^t(j) \geq h^t(j + 1)$.
Using more servers for a request cannot increase the cost of serving it.
\item The cost vector is convex.
This means that we have $h^t(j+1) - h^t(j +2) \leq h^t(j) - h^t(j + 1)$.
So the marginal improvement achieved by adding a new server to the request decreases with the number of servers already at the request.
\end{itemize}
When a request comes in, we can move an arbitrary number of available servers to the request before serving it.
The total cost incurred by the algorithm is divided into two parts:
\begin{itemize}
\item \emph{Move-Cost}, the cost of moving servers.
\item \emph{Hit-Cost}, the cost of serving the requests.
\end{itemize}

The main contribution of~\cite{bansal10:k-server} to the randomized $k$-server conjecture is an online algorithm for Allocation-C with the following guarantees (OPT is the optimal offline cost):
\begin{itemize}
\item The service cost is at most $(1+\epsilon)$ OPT.
\item The movement cost is at most $O(\log\frac{k}{\epsilon})$ OPT.
\end{itemize}
This is a relevant result because Cote et al.\ showed that an algorithm satisfying similar conditions for the more general \emph{Allocation problem} would yield a polylogarithmic algorithm for the randomized $k$-server problem~\cite{cote08:k-server}.


\subsection{The Caching with Costs Problem}
Instead of using the primal-dual approach directly, the authors first reduce the Allocation-C problem to another online problem.
In the \emph{Caching with Costs} problem, we have a uniform metric on $l$ points and up to $k$ available servers.
For each request $t$, the number of available servers is limited to $k(t) \leq k$.
Each request is described by a cost vector $c^t = (c_1^t, \ldots, c^t_l)$.
We can redistribute servers among the $l$ points when a new request comes in.
In contrast to the Allocation-C problem, we now incur a cost $c_i^t$ for each point $i$ with no server present.
The total cost incurred by the algorithm is divided into the movement cost and the hit cost.

The authors show that Allocation-C and Caching with Costs can be reduced to each other in an online fashion.
Moreover, Caching with Costs can be simplified so that at time $t$, only one location $p_t$ has non-zero cost $c_{p_t}$.
The primal LP for Caching with Costs is:
\[
\textrm{($P$) : min}  \sum_{t=1}^T c_{p_t, t} \cdot y_{p_t,t} + \sum_{i=1}^n\sum_{t=1}^{T+1} z_{i,t}
\]
\[
	\begin{array}{rr}
	\textrm{subject to :} & \\
		\forall t \textrm{ and } S \subseteq [l] \textrm{ :} & \sum_{i : i \in S} y_{i,t} \geq |S| - k(t) \\
		\forall t,i \textrm{ :} & z_{i,t} \geq y_{i,t-1} - y_{i,t} \\
		\forall t,i \textrm{ :} & z_{i,t}, y_{i,t} \geq 0 \\
	\end{array}
\]


The LP is similar to the one for the weighted caching problem described earlier.
The variable $y_{i,t}$ indicates whether a server is located at point $i$ at time $t$.
Since $z_{i,t} \geq y_{i,t-1} - y_{i,t}$, the variable $z_{i,t}$ denotes the cost for moving a server to point $i$ at time $t$.
So the first term of the objective captures the hit cost and the second term the movement cost.
The first constraint guarantees that at most $k(t)$ servers are in use at time $t$.

The dual LP is:
\[
\textrm{($D$) : max}  \sum_{t=1}^T\sum_S (|S| - k(t)) \alpha_{t,S} + \sum_{i | \textrm{no server at }i} \gamma_i
\]
\[
	\begin{array}{rr}
	\textrm{subject to :} & \\
		\forall t,i \textrm{ :} & \sum_{S: i\in S} \alpha_{t,S} + \beta_{i,t} - \beta_{i,t+1} \leq c_{i,t}\\
		\forall t,i \textrm{ :} & \gamma_i \geq \beta_{i, 1} \\
		\forall t,i \textrm{ :} & 0 \leq \beta_{i, t} \leq 1\\
		\forall t \textrm { and } S \subseteq [l] \textrm{ :} & a_{t,S} \geq 0\\
	\end{array}
\]


The algorithm for Caching with Costs is similar to weighted paging.
Again, we maintain a relation between the primal and dual:
\[
y_{i,t} \leftarrow \frac{\epsilon}{1 + k} \left( \exp\left( \ln\left( 1 + \frac{1+k}{\epsilon} \right) \cdot \beta_{i, t+1} \right) - 1 \right)
\]
By considering the derivatives of the primal and dual objective functions, the authors prove the desired bounds for Caching with Costs which carry through to Allocation-C.


\section{Conclusions}

\bibliographystyle{abbrv}
\bibliography{6.854-project}

\end{document}
