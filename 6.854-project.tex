\title{A Survey of Recent Work on Designing Competitive Online Algorithms via the Primal--Dual Approach}
\author{Fredrik Kjolstad \and Ludwig Schmidt \and Christos Tzamos}
\date{}

\documentclass[10pt, twocolumn]{article}

\usepackage{amsmath}
\usepackage{amsthm}
\usepackage{url}
\usepackage{relsize}
\usepackage{xspace}
\usepackage{subfigure}
\usepackage{graphicx,color}

\usepackage{fullpage}

% Todos
\newcommand{\fred}[1]{%
	\textcolor{red}{[#1]}
}
\newcommand{\christos}[1]{%
	\textcolor{red}{[#1]}
}
\newcommand{\ludwig}[1]{%
	\textcolor{red}{[#1]}
}



\begin{document}
\maketitle

\begin{abstract}
Online algorithms have become increasingly popular, partly because they capture uncertainty in a realistic way that makes them a useful model for practial applications.
The primal-dual method is an approach that has been gaining popularity, especially as a technique to arrive at approximations for NP-hard problems.
More recently it has also been applied as a general framework to solve many online algorithms.
In 2009, Buchbinder and Joseph published a survey of applications of the primal--dual method to online algorithms.
Since then new online problems have been tackled using this technique, such as the online node-weighted Steiner tree problem, the k-server problem, and online job-migration.
We present a survey of recent applications of the primal-dual method to online problems.
\end{abstract}

\section{Introduction}
% Short paragraph to tie intro to abstract
The primal-dual technique was successfully applied to approximate NP-Hard problems in the 1990's \fred{cite}.
In the early 2000's the application of the technique to online algorithms was discovered, and has since then been successfully used to solve many online algorithms \fred{cite}.
% Talk about how it allows you to specify online programs as offline linear programs that are then solved incrementally online using the technique.
% Solve program incrementally by successively adding in variables and constraints

% Buchbinder and Joseph's work
In 2009 Buchbinder and Naor published a survey of the application of the primal-dual method to online algorithms~\cite{buchbinder09:survey}.
The survey gave a thorough description of the technique, and went on to survey several papers showing how the primal-dual method can be applied to solve a number of important online problems. 
Examples include online set-cover, caching, routing and ad-auction revenue maximization.

Since then the primal-dual method has been applied to solve additional problems in an online setting.
In this paper we survey three new application of the technique to very different problems.
This demonstrates the power and applicability of the primal-dual approach to online problems.

In section~\ref{primal-dual} we offer an introduction to the primal-dual method using ski rental as a running example.
In section~\ref{steiner-tree} we survey Naor, Panigrahi and Singh's paper from 2011, which presents the first online algorithm for the node-weighted steiner tree problem~\cite{naor11:node-weighted-steiner-tree}.
\fred{Christos: Could you add one sentence on your paper here?}
In section~\ref{job-migration} we survey Buchbinder, Jain and Menache 2011 paper and companion technical report, in which they design an algorithm for online job-migration between geographically distributed cloud data centers to reduce electricity costs~\cite{buchbinder11:job-migration,buchbinder11:job-migration-techreport}.
This paper demonstrates the flexibility of the primal-dual technique, by applying it to the design of algorithms for complex real-world problems that beats reasonable greedy heuristics.
Finally, section~\ref{k-server} surveys Bansal, Buchbinder and Naor's 2010 paper that demonstrates the power of the primal-dual approach by using it to explore promising approaches to the online k-server problem, holy grail of online algorithms.
\fred{Ludwig: Could you look over the k-server sentence?}




\section{The Primal-Dual Approach}

In this section we present the primal-dual approach.
First, we explain the required background in linear programming and duality.
We then introduce the technique using the well known ski rental problem as an example.
Finally, we discuss the general primal--dual framework so that it can be applied to other problems.

We assume the reader is already familiar with the online algorithm model, competitive analysis, approximation algorithms, (integer) linear programming, randomized rounding and the concepts of weak and strong duality and will not review these concepts here.
For a brief discussion of these concepts we refer the reader to section~2 of Buchbinder and Naor's survey~\cite{buchbinder09:survey}.

\subsection{Preliminaries}
In addition to the standard theorems and definitions about linear programs and duality we introduce \emph{approximate} complementary slackness.
First, we review the concept of (ordinary) complementary slackness.
Consider the following linear programs:

\vspace{-.5cm}
\begin{align*}
& \textnormal{Primal: } \textnormal{minimize} \sum_{i=1}^n c_i x_i \\
 &\qquad \textnormal{subject to } \sum_{i=1}^n a_{ij} x_i  \geq b_j \, , \;\; x_i \geq 0 \\
& \textnormal{Dual: } \textnormal{maximize} \sum_{j=1}^m b_j y_j \\
 &\qquad \textnormal{subject to } \sum_{j=1}^m a_{ij} y_j \leq c_i \, , \;\; y_j \geq 0
\end{align*}
\vspace{-.5cm}

A feasible solution $x$ to the primal problem satisifies the complementary slackness condition if and only if for each $i$ we have either $x_i = 0$ or $\sum_{j} a_{ij}y_j = c_i$.
So either the variable $x_i$ is equal to 0 or the corresponding constraint in the dual is tight.
A feasible solution $x$ is optimal if and only if it satisifies the complementary slackness condition.

Approximate slackness relaxes the conditions above.
As a result, we only get an approximation ratio between the primal and dual solutions satisfying the approximate complementary slackness condition.
In the context of online algorithms, this ratio corresponds to the competitive ratio.

Now we precisely state the approximate complementary slackness condition.
Let $x$ and $y$ be feasible solutions to the primal and dual linear programs.
First we define the primal and dual complementary slackness conditions.

$x$ satisfies the \emph{primal} complementary slackness condition if for $\alpha > 1$ we have $x_i > 0$ or $c_i / \alpha \leq \sum_j a_{ij} y_j \leq c_i$ for any $i$.
Similarly, $y$ satisifies the \emph{dual} complementary slackness condition if for $\beta > 1$ we have $y_i > 0$ or $b_j \leq \sum_i a_{ij} x_i \leq b_j \beta$.

If $x$ and $y$ satisify the primal and dual complementary slackness conditions respectively then 
\[
\sum_{i=1}^n c_i x_i  \leq \alpha \beta  \sum_{j=1}^m b_j y_j
\]

Note that for $\alpha = \beta = 1$ this gives the orindary complementary slackness theorem.
The proof follows directly from applying the primal and dual complementary slackness properties to $\sum_i c_i x_i$.

\subsection{Ski Rental}

In this section we informally present the primal-dual approach by applying it to the ski rental problem.

In the ski rental problem a skier will go skiing several times in his life.
Every time he goes skiing he has to decide whether to rent a pair of skis or to buy a pair of skis that he can use for all subsequent ski trips.
Renting costs \$1, while buying skis costs \$B.
The goal of the skier is to spend the least amount of money.
The ski rental problem is interesting because the skier does not know beforehand how many days he will ski in his life --- after all he may break his leg tomorrow.
We assume an adversarial model where fate will ensure the worst possible outcome no matter what the skier decides.

\subsubsection{The Online Ski Rental Algorithm}
Against a malicious adversary, the optimal deterministic strategy that minimizes the competitive ratio has been known for a long time, and requires the skier to rent for $B$ days before buying.
If the last day of skiing is before day B then this strategy is optimal.
On the other hand, if the last day of skiing is after day B, then the strategy is at most two times as expensive as the optimal strategy, OPT, which is to buy skis on the first day of skiing.
This strategy achieves a competitive ratio of 2 (it is 2-competitive), and is optimal for deterministic strategies.

However, a better competitive ratio of $\frac{e}{e-1}$ can be achieved using randomization.
We will provide an optimal randomized algorithm using the primal-dual approach, but first we will discuss a deterministic primal-dual algorithm that achieves 2-competitiveness.

\subsubsection{LP Formulation}
We begin by formulating an integer linear program that captures the solution for the offline ski rental problem.
We use an indicator variable $x \in \{0,1\}$ that represents whether we buy the skis and a variable $d_i \in \{0,1\}$ for every day that indicates whether we rent skis at day $i$. 
The objective we want to minimize then is:
\[ B\cdot x + \sum^n_{i=1} d_i \]
subject to the constraints that for each day $i$:
\[ x + d_i \ge 1 \]

The optimal solution in an offline version is either $x=1$ and $d_i = 0$ for each day or $x=0$ and $d_i = 1$ for each day, whichever is best.
That is, the optimal solution is to buy at once or never buy at all.
This of course requires prior knowledge of $n$ which we do not have in an online setting.

In an online setting we know the objective beforehand, but we begin without any constraints.
Every day, a new constraint appears and we have to re-optimize our solution.
However, since we cannot change the past we cannot undo any previous decisions.
This means that we can never decrease any variables set in the past.

\subsubsection{Deterministic Algorithm}

We will now describe how we can incrementally solve the linear program using the primal-dual approach.
We will later argue that this will not cause the solution to deviate too much from OPT.
In order to apply the primal-dual approach, we first have to relax the integer linear program to a linear program. Note that the optimal solution in the relaxed version is exactly the same as before.

The primal-dual method approximates the solution to the linear program incrementally by updating the primal and dual simultaneously.
The primal and dual is given below.

\[
	\begin{array}{lr}
	\textrm{Primal: minimize}   & B\cdot x + \sum^n_{i=1} d_i   \\
	\textrm{subject to:} & \\
	\textrm{for each day $i$} & x + d_i  \ge 1  \\
			    & x     \geq 0, \forall i : d_i \ge 0
	\end{array}
\]
\[
	\begin{array}{lr}
	\textrm{Dual: maximize}   & \sum^n_{i=1} y_i   \\
	\textrm{subject to:} & \sum^n_{i=1} y_i \le B \\
	\textrm{for each day $i$} & 0 \le y_i  \le 1
	\end{array}
\]

For the online version, in the $i$-th day a new constraint ($x + d_i  \ge 1$) arrives in the primal problem (updating the objective as necessary) and a new variable ($y_i$) arrives in the dual.
With the new constraint the primal problem may become infeasible (this happens if $x < 1$), so we need to apply an update rule to decide which variable to increase.
If it is still feasible we do not have to do anything since we already covered the new constraint. 
Otherwise, we need to decide which variable, $x$ or $d_i$, to increase. 
To make this decision we look at the dual. 
Since the dual is a maximization problem, we would like to increase the new variable $y_i$ as much as possible.
Thus we increase $y_i$ until we hit a constraint in the dual problem.
This constraint corresponds to a variable in the primal problem, either $d_i$ or $x$.
We set this variable equal to 1.

Notice that the algorithm corresponds to exactly the same strategy we described previously, i.e. the skier rents skis for the first B days and then buys them.

%\subsubsection{Analysis}

The analysis of the ski rental problem is crucial for understanding the concepts of the primal-dual method. 
The key idea is that every time the algorithm makes a change in the primal LP, a change in the dual LP happens as well so that the ratio between their corresponding objective values remains bounded by some function. 
This function gives the competitive ratio of the algorithm.

For the previous algorithm, we note that the update rule always maintains a feasible solution for both the primal and the dual. 
In order to bound the ratio between their objectives we use approximate complementary slackness. 
By the update rule we have that:
\begin{itemize}
\item Whenever $x > 0$, $\sum^n_{i=1} y_i = B$ (tight)
\item Whenever $d_i > 0$, $y_i = 1$ (tight)
\item Whenever $y_i > 0$, $1 \le x+d_i \le 2$ (approximately tight)
\end{itemize}

So by approximate complementary slackness with $\alpha = 1$ and $\beta = 2$ we get that the primal objective is at most 2 times the dual objective. Therefore, it is at most 2 times the optimal offline solution and thus it is 2-competitive.

\subsubsection{Randomized Algorithm}
We now move on to get a randomized algorithm for the ski rental problem. 
We first try to find a fractional online solution and then use the fractional values of the primal variables to get the probabilities of buying and renting the skis for each day.
In the deterministic case we always maintained a feasible primal solution and whenever a new constraint appeared on day $i$ we either set $d_i$ or $x$ to 1. For the fractional case we do not have to set $x$ directly to 1 when we reach $B$ days but we can gradually increase $x$ every day. So for every new day $i$ if $x<1$ we use the update rule:
\begin{itemize}
\item $x \leftarrow x(1+1/B) + 1/(c\cdot B)$
\item $d_i \leftarrow 1 - x$
\item $y_i \leftarrow 1$
\end{itemize}
where $c$ is a constant to be defined later.

We move on to the analysis part of the algorithm. Let $x^{(i)}$ be the value of $x$ on the $i$-th day. We first note that under the update rule chosen, after $k$ days $x^{(k)} =  \sum_{i=0}^{k-1} (1+1/B)^i / (c\cdot B) = [(1+1/B)^k-1]/c$. It is obvious that the primal solution is always feasible since at any time $i$, $x^{(i)}+d_i=1$ and no previous constraints are violated by the increase in $x$. We choose $c$ so that the dual is feasible as well. We want $y_i$ to be 1 for at most $B$ days otherwise the dual constraint  $\sum^n_{i=1} y_i \le B$ will be violated. So we want that after $B$ days $x = 1$ which means that $c = (1+1/B)^k-1$.

Moreover, we want the ratio between the primal objective and the dual objective to be bounded by a constant.
We use a proof by induction to show that it is bounded by $(1+1/c)$.
Initially, the objective functions of the primal and the dual are both 0 so the statement holds.
Assuming that it holds for $k$ days, we now prove that it holds for $k+1$ days.
If $k\ge B$ it clearly holds since the two objectives are left unchanged.
Otherwise, we see that $x$ increases by $x^{(k)} / B + 1/(c \cdot B)$ while $d_{k+1}$ becomes $1 - x^{(k+1)}$.
So the primal objective increases by $x^{(k)} + 1/c + 1 - x^{(k+1)} \le 1 + 1/c$ since $x^{(k)} \le x^{(k+1)}$.
On the other hand the dual objective increases by 1 unit so the statement holds.

Therefore we get an online fractional algorithm for the ski rental problem with competitive ratio $1+1/c \approx e/(e-1)$ for large values of $B$.
Following we see how to make this randomized based on the fractional values we get.

%\subsubsection{Randomized Ski Rental}
Our randomized strategy for deciding is as follows: In the beginning we pick a random value $v$ in the interval $[0,1]$ and we buy skis on the first day that $x \ge v$.
The probability of buying skis in the first $i$ days is exactly $x^{(i)}$, i.e. the assigned fractional value by the online fractional solution. The probability of renting skis in the $i$-th day is exactly $1-x^{(i)}$.
If we go skiing for $n$ days the expected cost of our randomized algorithm is therefore $B \cdot  x^{(n)} + \sum_{i=1}^n(1-x^{(i)})$ which is exactly the same as the objective of the fractional algorithm.
Thus the randomized algorithm is also $e/(e-1)$-competitive.

\subsection{General Approach}
\label{section:general_approach}
We can generalize the approach used for ski rental to a wide class of linear programs.
This class contains linear programs of the following form:
\[
	\begin{array}{lr}
	\textrm{Primal: minimize}   & \sum_{i=1}^n c_i x_i   \\
	\textrm{subject to:} & \\
	\textrm{for $1 \le j \le m$} & \sum_{i \in S(j)} x_i  \ge 1  \\
	\textrm{for $1 \le i \le n$} & x_i  \ge 0  \\
	\end{array}
\]
where each $S(j)$ is an arbitrary subset of the variables.
Such LPs are called \emph{covering LPs}, because they model problems where all elements of a set have to be covered by a collection of other sets.
Examples of this include ski rental, where all days have to be covered by renting or buying skis, vertex cover, edge cover, set cover, and the problems in section 3--5.

The dual of a covering LP is called a \emph{packing LP} and has the following form:
\[
	\begin{array}{lr}
	\textrm{Dual: maximize}   & \sum_{j=1}^m y_j   \\
	\textrm{subject to:} & \\
	\textrm{for $1 \le i \le n$} & \sum_{j | i \in S(j)} y_j  \le c_i  \\
	\textrm{for $1 \le j \le m$} & y_j \ge 0  \\
	\end{array}
\]

For instance, the matching problem is the packing dual of the vertex covering primal, and the independent set is the packing dual of the edge cover covering primal.

As the reader might see this duality nicely captures ski rental.
The ski rental primal is a covering problem where we must cover all days with rented or bought skis, and the corresponding dual is a packing problem, where we pack the cost of renting skis int the cost $B$ of buying skis.

In the online version a constraint appears every time in the primal LP and a new variable arrives in the dual. We are only allowed to increase the variables at every time step.

Algorithm \ref{generalalg} provides an online fractional solution for every such problem. Using a similar analysis as in the ski-rental problem we can prove that the algorithm is $O(\log d)$-competitive, where \mbox{$d = \max_j\{|S_j|\}$}~\cite{buchbinder09:survey}.
Note that the algorithm does not depend on the dual variables as in the randomized ski rental version. The dual variables are only used in order to bound its competitiveness by comparing the primal and the dual objectives.
Using this general algorithm we can develop online algorithms for many different optimization problems as we will see in later sections. This algorithm gives us out of the box an efficient online fractional solution as compared to the optimal. It is our responsibility however to use the fractional values that we get to design our response.

\begin{algorithm}
\caption{Whenever a new constraint in the primal and a variable in the dual appear}
\label{generalalg}
\begin{algorithmic}[1]
\WHILE{ $\sum_{i \in S(j)} x_i  < 1$}
  \FORALL{$i \in S(j)$}
    \STATE $x_i \leftarrow x_i (1 + 1/c_i) + 1/(|S(j)| \cdot c_i)$ 
  \ENDFOR
  \STATE $y_j \leftarrow y_j + 1$
\ENDWHILE
\end{algorithmic}
\end{algorithm}





\section{Online Node-weighted Steiner Tree}
In this section we explore several applications of the primal-dual framework to several online network optimization problems. We begin by looking at a very general problem called the generalized connectivity problem. In this problem there is an underlying undirected graph $G = (V,E)$ where each edge has a non-negative cost $c_e$. At each time step, a new request arrives in the form $(S_i,T_i)$ where $S_i$ and $T_i$ are two non-empty disjoint subsets of $V$ and the goal is to connect some node in $S_i$ to some node in $T_i$ via a path. For every request, we need to decide which edges to buy so that we can satisfy it. Note that edges that were previously purchased can be reused for handling future requests. The goal is to minimize the total cost of purchased edges.

\subsection{Fractional Connectivity Problem}

We are interested in solving the fractional version of the connectivity problem where we are allowed to buy fractions of the edges (capacity) and we only require that a unit of flow can be sent from nodes in $S_i$ to nodes in $T_i$. In the online setting once we bought some capacity in an edge we can never decrease it but we may increase it later on. To approach this we will apply the framework that was previously described. We first formulate the LP where for each edge we keep a variable $x_e$ indicating the capacity of the edge that we own. We note that a unit flow can be sent from $S_i$ to $T_i$ if and only if every $(S_i, T_i)$ cut with respect to $x_e$ has value at least 1. Therefore, the LP we get is: 
\[
	\begin{array}{lr}
	\textrm{Primal: minimize}   & \sum_{e \in E} c_e x_e   \\
	\textrm{subject to:} & \\
	\textrm{for each day $i$, every $(S_i,T_i)$ cut $C$} & \sum_{e \in C} x_e  \ge 1  \\
			    & x_e \geq 0
	\end{array}
\]

To get an online solution for the LP we use the framework developed in the previous section.The difference is that now instead of one constraint we get an exponential number of constraints added per request. We deal with this however by finding the constraint that corresponds to the minimum $(S_i,T_i)$ cut every time and increase the primal and the dual functions as mentioned in the previous section. So in total the algorithm does the following for every request:

\begin{algorithm}
\caption{Request($S_i$,$T_i$)}
\begin{algorithmic}[1]
\WHILE{the max $S_i \rightarrow T_i$ flow is less than 1}
  \STATE Find the minimum  $(S_i,T_i)$ cut $C$
  \FORALL{edges $e \in C$}
    \STATE $x_e \leftarrow x_e (1 + 1/c_e) + 1/(|C| \cdot c_e)$ 
  \ENDFOR
  \STATE $y_{i,C} \leftarrow y_{i,C} + 1$
\ENDWHILE
\STATE Make each edge $e$ have capacity $x_e$
\end{algorithmic}
\end{algorithm}

The algorithm above has competitive ratio $O( \log \textrm{MAX-CUT} )$ and runs in polynomial time.

\subsection{Facility Location}
We now move on to develop a randomized strategy for the integral version of the connectivity problem based on the fractional solution we got above.Since this is in general very hard to do we will restrict ourselves in a very interesting subproblem, the (non-metric) facility location.

In this problem, there is a set $F$ of possible facilities that we can build along with a certain building cost $c_f$ for every $f \in F$. In this problem clients arrive one at a time and want to be connected to at least one of the built facilities. Every client $i$ has a certain cost $c_{i,f}$ for every facility $f$ that he may be connected to. So the decision we have to make whenever a client appears is whether to connect him to some previously built facility or to build a new facility and connect him to it. Again, since this is an online setting we cannot undo the fact that a facility has been built or that a client has been connected to the facility.

We now show how to reduce this problem to the connectivity problem. For every facility $f \in F$, we create a tree where the root corresponds to $f$. We connect the root to a single node with an edge of cost $c_f$. And then we connect this node to the leaves where each leaf corresponds to every client. The edges connecting the client nodes to the intermediate node have cost equal to $c_{i,f}$. Note that each tree has unbounded size since the number of clients may be arbitrarily large and that in the online setting we don't know the edge costs of a client before he appears. However we will never use a part of the tree that corresponds to some client that hasn't arrived yet.

Every time a new client appears we need to connect a node that corresponds to him in some tree to the root of that tree by buying edges. If the edge connecting the root of the tree to the intermediate node has already been bought it means that the facility has been built and that we only need to pay for connecting the client to that facility (buy the edge to the intermediate node). Thus, in the context of the connectivity problem we can see that each client is associated with two sets $(S_i,T_i)$ where $S_i$ is the set of nodes corresponding to him in every tree and $T_i$ are the roots of all trees.

Therefore we can use the fractional algorithm we developed previously to get a fractional solution to the problem. Since the cut every time is exactly equal to $m$ where $m$ is the number of trees/possible facility locations the competitive ratio is $O(\log m)$. We now see how we can use the fractional solution to obtain an online randomized algorithm for facility location.

Throughout the course of the algorithm, we maintain $2 \lceil log(n'+1) \rceil$ values for every tree when $n'$ requests have occurred so far. This values are drawn uniformly at random from $[0,1]$. Before any requests occur no tree has any such value. Whenever a request happens we sample more values so that every tree has exactly $2 \lceil log(n'+1) \rceil$. In order to decide which edges to buy we compute for every tree the minimum of its values and buy all edges with fractional capacity greater than this value. For a single value drawn randomly out of $[0,1]$ edge $e$ has probability $x_e$ of being selected. By the union bound for $2 \lceil log(n'+1) \rceil$ values the edge has probability less than $2 \lceil log(n'+1) \rceil x_e$ of being selected. So in total the expected cost is at most $2 \lceil log(n'+1) \rceil = O(\log n')$ times the cost of the fractional solution. This means that it is at most $O(\log n \log m)$ times the optimal offline algorithm.

The resulting solution however may not be feasible since it may be the case that a client is not connected to any facility.
This happens with low probability: $1/n'^{2}$~\cite{naor11:node-weighted-steiner-tree}.
We can therefore connect client $i$ to some root via the cheapest path. This is a lower bound to the value of the optimal offline solution since OPT has to connect client $i$ to some facility. So overall the additional expected cost added when the randomization fails to connect a client to a facility is $\sum_{n'} OPT/n'^2 = O(1) OPT$ which is negligible.

So the competitive ratio of the randomized algorithm is $O(\log n \log m)$ for a total of $n$ requests.

\subsection{Node-Weighted Steiner Tree}

In the node weighted Steiner tree problem, there is an underlying graph $G=(V,E)$ where both vertices and edges have costs ($c_v$ and $c_e$) respectively.At all times, we maintain a network of connected nodes in the graph. Initially this network is empty. Every time a request arrives on a vertex of the graph and it has to be connected to the network via a path formed by nodes and edges that we buy. Again we need to decide every time which nodes and edges to buy to minimize the cost of keeping all request points connected. Once we buy a node or an edge we can never undo this action.

This problem is very similar to the previous problems and in fact we will show how we can reduce it to facility location to obtain an online algorithm. To do this we use the following lemma ~\cite{naor11:node-weighted-steiner-tree}:

\begin{lemma}
For any set of nodes $S = \{s_1,...,s_k\}$ and any tree $T \subseteq G$ that contains all nodes in $S$, 
there exist nodes $v_2, v_3, ... , v_k \in T$, not necessarily distinct, for which:
\[ \sum_{i=2}^k [c(P_i) - c_{v_i}] \le O(\log k) c(T) \]
where $c(P_i)$ is the cost of the cheapest path that connects node $s_i$ to some previous node $s_j$ with $j<i$ while going through $v_i$ and $c(T)$ is the cost of the tree. Costs include both node and edge costs.
\end{lemma}

Using this lemma we observe that for the optimal Steiner tree $T^*$ and its corresponding $v_2, v_3, ... , v_k \in T^*$ it holds that:
\[ \sum_{i=2}^k [c(P_i) - c_{v_i}] + \sum_{v \in \{v_2, v_3, ... , v_k\}} c_{v} \le O(\log k) c(T^*) \]
where the second summation is over the distinct nodes $v_2, v_3, ... , v_k$ and it is at most $c(T^*)$.

It follows that the quantity that minimizes $\sum_{i=2}^k [c(P_i) - c_{v_i}] + \sum_{v \in \{v_2, v_3, ... , v_k\}} c_{v}$ for some nodes $v_2,...,v_k$ is $O(\log k)$ approximate for the node weighted Steiner tree problem. This problem can be viewed as a facility location problem where nodes in the graph are the candidate facility locations and the chosen $v_i$ correspond to the built facilities. Every time a request comes at a node $v$ we calculate the minimum cost of connecting it to some previously requested node for every node $i$ in the graph without including the cost for node $i$. Viewed as a facility location problem this cost corresponds to the cost of connecting the new client to facility $i$. Using the previously developed algorithm for facility location we get a $O(\log^2 k \log n)$ where $k$ is the number of requests and $n$ is the number of nodes in the graph. 



\section{Online Job-Migration}
In this section we demonstrate the applicability of the primal-dual method to real-world problems.
We survey a systems paper, which is primarily interesting because of its applicability to systems design in the real-world, and not necessarily because of its theoretical value.

Linear programs are able to capture many real world problems, and the primal-dual technique gives us a tool that lets us to design online algorithms from many linear programs.
We we illustrate this here by discussing its applicability to a data center job-migration problem.

The goal of the job migration is to move jobs to data centers at locations with temporarily cheaper power, to decrease energy cost since it now dominate data center operation costs.
An online algorithm based on the primal-dual method, that takes both energy costs and bandwidth costs into account, was developed by Buchbinder, Jain and Menache~\cite{buchbinder11:job-migration,buchbinder11:job-migration-techreport}.
The algorithm is $O(\log H_0)$-competitive, where $H_0$ is the total number of servers in all data center.

The algorithm developed by the authors using the primal-dual method turns out to be too computationally expensive to be used in practice.
However, it serves as the inspiration for a non-intuitive algorithm that they empirically show to outperform two reasonable greedy heuristics and to be within $4\%-6\%$ of the optimal, on real-world data.

\subsection{Problem Formulation}

A technique that has recently been proposed to reduce energy costs when operating multiple geographically distributed data centers is to move tasks to data centers with cheaper energy.
This approach is motivated by the observation that energy prices are localized and that computation will therefore be cheaper at some sites.
Furthermore, energy prices are fluctuating and the cheapest data center at one point in time may not be the cheapest later on.
This motivates online algorithms that monitor electricity prices and at any time move tasks to data centers where the electricity is the cheapest.
However, moving data is not free and a cost proportional to the amount of data used is typically incurred.
These competing constraints make the problem non-trivial, and greedy heuristics with provable performance are not feasible.

% Describe the model and the variables
Specifically, our problem consist of $n$ data centers, where each data center $i$ contains $H_i$ servers, with $H_0$ denoting the total number of servers in all data centers.
The job-load in the system is given by the constant variable $B$ (in the technical report the authors also generalizes to variable workloads).
At each discretized time step $t$ we must decide whether to move jobs between data centers, taking the bandwidth cost and energy costs into account.
The migration cost for moving data out of data center $i$ is defined as $d_i$, and the variable $z_{i,j,t}$ is the number of jobs migrated from server $j$ in data center $i$ at time $t$.
Finally, $c_{i,j,t}$ is the energy cost of operating server $j$ in data center $i$ at time $t$, and $y_{i,j,t}$ is the fraction of server $j$ of data center $i$ that is utilized at time $t$.
Jobs are defined as using exactly one server, so the number of servers needed at data center $i$ at time $t$ is $\lceil \sum^{H_i}_{j=1}y_{i,j,t} \rceil$.
Since the number of servers in each data center is expected to be large we can ignore the extra fraction introduced by the ceiling operator.

% Define the LP
The following (primal) linear program captures the online data center job-migration problem:
\[
\textrm{($P$) : min}  \sum^n_{i=1}\sum^{H_i}_{j=1}\sum^{T}_{t=1}d_{i} \cdot z_{i,j,t} + \sum^{n}_{i=1}\sum^{H_{i}}_{j=1}\sum^{T}_{t=1} c_{i,j,t} \cdot y_{i,j,t}
\]
\[
	\begin{array}{rc}
	\textrm{subject to :} & \\
		\forall t \textrm{ :} & \sum^n_{i=1}\sum^{H_i}_{j=1}y_{i,j,t} \geq B \\
		\forall i,j,t \textrm{ :}	    & z_{i,j,t} \geq y_{i,j,t-1} - y_{i,j,t} \\
		\forall i,j,t \textrm{ :}	    & y_{i,j,t} \leq 1 \\
		\forall i,j,t \textrm{ :}	    & y_{i,j,t} \geq 0 \\
		\forall i,j,t \textrm{ :}	    & z_{i,j,y} \geq 0
	\end{array}
\]

The objective function minimizes the sum of the bandwidth cost (first term) and the energy cost (second term).
The first constraint ensures the allocation satisfies the workload $B$.
The second constraint enforces that the amount of work leaving a server is at least the same as the work that server lost (it may be more as a server may also receive work).
The third prevents servers from being over-used as each server can only execute one task, and the fourth and fifth constraints just prevents server usage or bandwidth from being negative.

The corresponding dual linear program is:

\[
\textrm{($D$) : max}   \sum^{T}_{t=1}B \cdot a_t - \sum^{n}_{i=1}\sum^{H_i}_{j=1}\sum^{T}_{t=1}s_{i,j,t}
\]
\[
	\begin{array}{rc}
	\textrm{s.t. :} & \\
		\forall i,j,t \textrm{ :}	    & -c_{i,j,t} + a_t + b_{i,j,t} - b_{i,j,t+1} - s_{i,j,t} \leq 0 \\
		\forall i,j,t \textrm{ :}	    & b_{i,j,t} \leq d_i \\
		\forall i,j,t \textrm{ :}	    & b_{i,j,t} \geq 0 \\
		\forall i,j,t \textrm{ :}	    & s_{i,j,t} \geq 0
	\end{array}
\]

\subsection{Primal-Dual Online Algorithm}
\label{pd-alg}

Given these  linear programs we can devise an online algorithm based on the primal-dual method.
That is, we simultaneously update the primal ($P$) and dual ($D$) in such a way that we always maintain a feasible solution to the dual.
We can then upper bound the operation cost of the dual, which by weak duality becomes the lower bound of the primal, and hence a bound on the competitive ratio of the algorithm.

Algorithm~\ref{job-migration-alg} shows update rule for updating the dual variables.
The update rule is applied for each server $(i,j)$, and the input of the algorithm is $i$, $j$, and the cost vector $c_t$. 
The cost vector $c_t$ is defined to be an \emph{elementary} cost vector, namely one where every entry is $0$, except for the one pertaining to the server $i,j$ at that execution of the update rule.
Note that a more precise name would be $c_{i_t,j_t}$, but $c_t$ is used instead for notational convenience.
Note also, that the reduction to elementary cost vectors does not affect our solution.

\begin{algorithm}
\caption{Procedure to update dual variables. The procedure is executed for every server $(i,j)$}
\label{job-migration-alg}
\begin{algorithmic}[1]
\REQUIRE Data-center $i_t$, server $j_t$, cost $c_t$ (at time $t$)

\WHILE{$y_{i_t, j_t, t} \neq 0$ \& $-c_t + a_t + b_{i_t,j_t,t} - b_{i_t,j_t,t+1} \neq 0$}
 \STATE Increase $a_t$ by a small amount
 \FORALL {$i,j \neq i_t,j_t$ such that $y_{i,j,t} \leq 1$}
  \STATE Increase $b_{i,j,t+1}$ with rate $\frac{d \cdot b_{i,j,t+1}}{d \cdot a_t} = 1$
 \ENDFOR
 \FORALL {$i,j \neq i_t,j_t$ such that $y_{i,j,t} = 1$}
  \STATE Increase $s_{i,j,t}$ with rate $\frac{d \cdot s_{i,j,t}}{d \cdot a_t}=1$
 \ENDFOR
 \FORALL {$i,j=i_t,j_t$}
  \STATE Decrease $b_{i_t,j_t,t+1}$ so that $\sum_{i,j \neq i_t,j_t}\frac{d \cdot y_{i,j,t}}{d \cdot a_t} = -\frac{d \cdot y_{i_t,j_t,t}}{d \cdot a_t}$
 \ENDFOR
\ENDWHILE
\end{algorithmic}
\end{algorithm}

The objective of the dual algorithm is to maximize the dual profit, and it is therefore beneficial to increase $a_t$ as much as possible, without violating constraints.
However, the update rule is expressed in terms of continuous updates to $a_t$ (i.e., update $a_t$ by a small amount).
In practice the algorithm would be implemented in terms of discrete-time iterations, and it is possible to achieve this by searching for a scalar value of $a_t$ up to the required precision level.

Finally, as stated earlier, the primal variable, $y_{i,j,t}$, is updated as well along side the updates to the corresponding dual variable $b_{i_t,j_t,t+1}$.
The following update rule is applied to the primal:
\[
 y_{i,j,t} \leftarrow \frac{1}{H_0}\left(\textrm{exp}\left(\ln(1+H_0) \frac{b_{i,j,t+1}}{d_i}\right)-1\right)
\]

Note that the process is driven by dual updates, and that we determine termination based on it.
The primal is tied to the dual by (weak) duality, and we can therefore update the primal variables based on their corresponding dual variable.

\begin{theorem}
 The online algorithm is $O(\log(H_0))$-competitive.
\end{theorem}

We do not provide a proof here, but it is based on lower bounding the value of the primal minimization problem by the value of the dual maximization problem, by taking advantage of these being linked through the primal update rule.
A proof sketch is given in~\cite{buchbinder11:job-migration-techreport}, and a detailed proof is given in~\cite{buchbinder11:job-migration}.

\subsection{Efficient Online Algorithm}

As we mentioned at the beginning of this section, the primal-dual algorithm in section~\ref{pd-alg} is too computationally expensive for practical usage.
A couple of reasons for this is the cost of discrete time steps in the dual update rule as opposed to a one-shot solution, and the cost of managing one elementary cost vector per dual variable.
Because of this and because this is an algorithm one would want to implement for data center management, the designers of the primal-dual algorithm also designed a practical $O(n^2)$ algorithm.
The algorithm is based on the primal-dual algorithm and uses the same ideas, but looses the provable competitive ratio in exchange being computationally less expensive.
To evaluate the algorithm the authors performed empirical experiments, comparing the algorithm with reasonable and much more intuitive heuristics on real energy cost data.

The iterations of the efficient algorithm is as follows. For each iteration the update rule specifies how much work to move from DC $i$ to DC $k$. Please refer to~\cite{buchbinder11:job-migration-techreport} for a detailed description:

\begin{algorithm}
\caption{Efficient job migration algorithm}
\label{job-migration-alg}
\begin{algorithmic}[1]
\REQUIRE The cost vector $c=(c_1,c_2,...,c_n)$, and the current load vector $y=(y_1,y_2,...,y_n)$
\FOR {$k \leftarrow i...n$}
 \FOR {$i \leftarrow k+1...n$}
  \STATE Move load from DC $i$ to DC $k$ according to the following migration rule 
   \[ 
    \min \left\{ y_i, H_k - y_{k}, s_1 \cdot \frac{c_{i,k}y_i}{d_{i,k}} \cdot (y_k + s_2) \right\}
   \]
   where $s_1,s_2 > 0$
 \ENDFOR
\ENDFOR
\end{algorithmic}
\end{algorithm}

% Experiments
To evaluate the algorithm the authors compared it to the following heuristics:
\begin{description}
 \item[Move to Cheap] At each time step, move all tasks to the servers with the least expensive power. This fully optimizes power usage, but does not optimize for bandwidth at all.
 \item[MimicOPT] At each time step, heuristically attempt to mimic OPT by solving an LP with the energy costs uo until the current time $t$. Note that this algorithm, as opposed to the primal-dual based algorithm, makes no attempt to take future power costs into account.
\end{description}

They ran the simulated experiments comparing the efficient primal-dual-based online algorithm, move to cheap, and mimicOPT to OPT.
The experiments were performed on publicly available electricity pricing information from 30 US locations covering January 2006 through March 2009.
The results show that the efficient primal-dual-based algorithm comes performs within $5.7\%$ of OPT, while move to cheap and mimicOPT only performs within $68.5\%$ and $10.4\%$ of OPT respectively.
This clearly motivates using an algorithm that is based an algorithm developed using the primal-dual method for such complex problems, as opposed to greedy heuristics even if these are more intuitive.





\section{Towards the randomized k-server conjecture: a primal-dual approach}
Recently, the primal-dual technique has been used to explore new approaches to the $k$-server problem~\cite{bansal10:k-server}.
In this application, the authors also extend the primal-dual technique to deal with a wider range of LPs.
We first demonstrate the extended primal-dual technique with an example from~\cite{bansal10:k-server}.
Then we briefly give an overview of the steps involved in applying the primal-dual technique to the $k$-server problem.

\subsection{Extended Primal-Dual Technique}
One of the main limitations of the original primal-dual technique is the restriction to non-negative constraints and variables.
Using the basic approach described in section \ref{section:general_approach}, it impossible to include natural constraints like $x_i \geq x_j$, which have to be converted to $x_i - x_j \geq 0$ in an LP (note the negative coefficient of $x_j$).
This condition restricts the set of problems we can model or makes natural LP formulations more complex.
The main technical contribution of~\cite{bansal10:k-server} is to overcome these limitations.

We illustrate the the extended primal-dual framework with the weighted caching problem.
In this problem, we have a cache of size $k$ and a total of $n$ pages, each with a weight $w_i$.
At each time step, we receive a request for a page $p$.
If $p$ is not in the cache, we have to evict one page from the cache and replace it with $p$.
Fetching $p$ into the cache costs $w_p$.
The goal is to minimize the total cost of fetching pages.

The first randomized $O(\log k)$ competitive algorithm for the weighted caching problem also used the primal-dual framework but required a more complicated approach than the following algorithm~\cite{bansal07:weighted-paging}.

\subsection{The $k$-Server Problem}
The problem statement is as follows:
We have $k$ servers located at up to $k$ points in a metric space (essentially a set with a well-behaved distance function) and must serve requests appearing one after another.
Each request is a point in the metric space and we serve it by moving one of the $k$ servers to the request.
The goal is to minimizhee the total distance travelled by all servers.

The $k$-server problem is one of the most important problems in the field of online algorithms.
One reason is that many online problems can be reduced to the $k$-server problem.
A common example is the paging problem. For $n$ pages and cache size $k$, the problem can be modelled by a uniform metric space on $n$ points and $k$ servers.
In a uniform metric space, the distance between all points is equal to 1.

It is conjectured that there is a deterministic $k$-competitive algorithm for the $k$-server problem.
While the best algorithm for the general case is $2k -1$ competitive, there are $k$-competitive algorithms for some special metrics like trees.
For the randomized case, the conjectured competitive ratio is $O(\log k)$ and there has been recent work showing a $\tilde{O}(log^3 n \; log^2 k)$ approximation ratio~\cite{bansal11:randomized-k-server}.
While the results in \cite{bansal11:randomized-k-server} do not use the primal-dual approach for online algorithms, some of the techniques are inspired by problems that were first introduced in the context of the primal dual approach.

\subsection{Applying the Primal-Dual Approach to the $k$-Server Problem}
Due to the importance of the $k$-server conjecture, there is already a large body of work on tackling the problem.
In order to put the primal-dual approach to the $k$-server problem in context, we review the high-level ideas here.

One key difficulty of the $k$-server problem on general metric spaces is the weak structure of the distances between points.
The most promising approach for overcoming this obstacle is work on embedding general metrics into hierarchically separated trees (HST).
In an HST, the distance between a node $x$ and its parent $p(x)$ depends only on a constant $\alpha$ and the depth $i$: $d(x, p(x)) = \alpha^i$.
Moreover, all leaf nodes in an HST are at the same level.
When embedding a metric space into an HST, only the leaves of the HST correspond to points in the metric space.
General metrics can be embedded into distributions over HSTs with only logarithmic distortion of the distances.
This distortion would be still be sufficient to provide polylogarithmic competitive ratios for the randomized $k$-server problem.

Cote et al.\ use HSTs introduce an approach to the $k$-server problem based on recursively solving a simpler problem at each node of the tree.
Due to the structure of HSTs, we have a uniform metric space on the children of a given node (all children have the same distance to their common parent).
This approach seems promising because of the relatively simple structure of uniform metrics.
Moreover, there has been recent success using this technique to derive algorithms for a closely related problem (metrical task systems).
In the case of $k$-server, the problem we want to solve at each node of the tree is the \emph{Allocation problem}.

The Allocation problem is defined as follows:
We have a uniform metric space with $n$ points and up to $k$ servers.
For each request $t$, the number of available servers is limited to $k(t) \leq k$.
Each request is described by a cost vector $h^t = (h^t(0), \ldots, h^t(k))$ where $h^t(j)$ is the cost of serving the request $t$ with $j$ servers.
The cost vectors satisfy a natural property: the costs are monotonic, i.e.\ $h^t(j) \geq h^t(j + 1)$ (using more servers for a requests cannot increase the cost of serving it).
When a request comes in, we can move an arbitrary number of available servers to the request before serving it.
The total cost incurred by the algorithm is divided into two parts:
\begin{itemize}
\item \emph{Move-Cost}, the cost of moving servers.
\item \emph{Hit-Cost}, the cost of serving the requests.
\end{itemize}

Cote et al.\ showed that an algorithm for the Allocation problem satisfying the following conditions leads to a polylogarithmic algorithm for the randomized $k$-server problem.
Let OPT be the cost of the optimal algorithm for the Allocation problem.
Then the conditions are as follows:
\begin{itemize}
\item The total Hit-Cost is at most $(1 + \epsilon)$ OPT.
\item The total Move-Cost is at most polylog $\cdot$ OPT.
\end{itemize}
However, Cote et al.\ only showed how to solve the Allocation Problem with these bounds for a metric space consisting of two points.
The authors of~\cite{bansal10:k-server} used the primal-dual approach to find an algorithm for a restricted version of the Allocation Problem on uniform metrics.

\subsection{The Allocation-C Problem}
The Allocation-C problem introduces a second constraint on the cost vector $h^t$.
In addition to being monotonic, the cost vector is now also \emph{convex}.
This means that we have $h^t(j+1) - h^t(j +2) \leq h^t(j) - h^t(j + 1)$.
So the marginal improvement in Hit-Cost for adding a new server to the request decreases with the number of servers already at the request.
While convexity seems to be a natural property, it does not arise from the reduction of Cote et al.
However, the authors of~\cite{bansal10:k-server} claim that the convexity property might hold in a more aggregate sense.

The main result of~\cite{bansal10:k-server} is an online algorithm for Allocation-C with the following guarantees (OPT is the optimal offline cost):
\begin{itemize}
\item The service cost is at most $(1+\epsilon)$ OPT.
\item The movement cost is at most $O(\log\frac{k}{\epsilon})$ OPT.
\end{itemize}

Instead of using the primal dual approach from \cite{buchbinder09:survey} directly, the authors first reduce the Allocation-C problem to another online problem.

In the \emph{Caching with Costs} problem, we have a uniform metric on $l$ points and up to $k$ available servers.
For each request $t$, the number of available servers is limited to $k(t) \leq k$.
Each request is described by a cost vector $c^t = (c_1^t, \ldots, c^t_l)$.
We can redistribute servers among the $l$ points when a new request comes in.
In contrast to the Allocation-C problem, we now incur a cost $c_i^t$ for each point $i$ with no server present.
The total cost incurred by the algorithm is divided into the movement cost and the hit cost.

In order to reduce the Allocation-C problem with $n$ points to an instance of Caching with Costs, we introduce $k$ points in Caching with Costs for each original point in Allocation-C (so $l = nk$).
This allows us to model assigning up to $k$ servers to a point $i$ in Allocation-C by assigning servers to the $k$ points corresponding to $i$ in Caching with Costs (note that in Caching with Costs, it is only relevant whether a point is being served, not by how many servers).
Given a request $h^t$ to point $i$ in Allocation-C, we translate it into Caching with Costs accordingly: $c^t_{i,j} = (h^t_{i,j-1} - h_{i,j}^t)$.
This reduction shows that the hit and movement costs incurred in Caching with Costs are equal to those incurred in Allocation-C.
Conversely, we can also show how to reduce Caching with Costs to Allocation-C.
Hence a $c$-competitive algorithm for one problem implies a $c$-competitive algorithm for the other problem.

Before stating the linear program for Caching with Costs, we need one further lemma.
TODO: state request decomposition lemma
TODO: state LP



\section{Conclusions}

\bibliographystyle{abbrv}
\bibliography{6.854-project}

\end{document}
